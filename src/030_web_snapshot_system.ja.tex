\chapter{Web-Snapshot システム}
\label{chap:web_snapshot_system}
本章では、コンテンツの閲覧状態を保存できるブックマークシステム、Web-Snapshotを提案する。
はじめにWeb-Snapshotシステムの概要を述べ、次に本研究における復元の定義を示す。
ついでWeb-Snapshotの特徴を説明する。
そして最後に、ユーザがWeb-Snapshotを利用する流れについて述べる。

\section{システムの概要}
Web-Snapshotはコンテンツの閲覧状態を保存できるブックマークアプリケーションである。
Webコンテンツをブックマークすると、URLとともにスクロール位置・動画再生位置・PDFのページ数などの情報を保存する。
保存されたブックマークはiOSアプリケーションおよびChrome拡張機能から閲覧・管理できる。
ブックマークしたコンテンツを再度開くと、保存時の閲覧状態を復元する。

\section{本研究における復元の定義}
\label{chap:web-snapshot-system-restore-definition}
本研究における『復元』の定義は、”保存時と同じスクロール位置でWebページを開くこと”とする。
本研究において、『スクロール位置』は、Webページ内のどのコンテンツが画面の左上に表示されているか、によって判定する。
なお、本研究における『画面』とはブラウザの画面のうち本文表示領域のことを指す。
すなわち、ウィンドウサイズ・画面の向き・ブラウザが変わったとしても、画面左上に同じコンテンツが表示される。
なお、保存時のピンチズームスケール\footnote{いわゆるズーム率。ピンチイン・アウトの動作によって画面を拡大・縮小する}やピクセル比\footnote{ディスプレイの物理ピクセルとデバイスに依存しないピクセルの比率}は考慮しない。
これは上記の要素はブラウザ上でread-only\cite{}であるため、本システムから設定できないためである。
復元の際は、上記の要素がデフォルトの設定であると仮定して復元する。
また、広告ブロッカーなどの拡張機能はすべて無効であると仮定する。


具体的には、ピンチズームスケールは100\%、ピクセル比は200\%のWebページの状態を復元する。
動画および音声については、上記に加えて保存時と同じ再生位置から再生が始まることを条件とする。
ただし、復元する再生位置の単位は秒とする。
例外として、PDFについては保存時と同じページ数が開くことを復元と定義する。

本研究における復元の定義を以下の表\ref{tb:intro-restore-definition}にまとめる。
本研究において、復元しない要素とその理由を表\ref{tb:intro-not-restore-definition}に示す。

% textlint-disable
\begin{table}[htbp]
  \begin{center}
    \caption{本研究における閲覧状態の復元の定義}
    \label{tb:intro-restore-definition}
    \begin{tabular}{|c|l|l|l|}
      \hline
      復元する項目 & 条件 &  復元の定義 \\\hline\hline
      スクロール位置 & 全て & 
        \begin{tabular}{l}
          保存時の画面左上に位置しているコンテンツが\\画面左上に表示されている。 
        \end{tabular}\\\hline
      動画再生位置 & 動画 & 保存時の再生位置(秒単位)から動画がはじまる。 \\\hline
      音声再生位置 & 音声 & 保存時の再生位置(秒単位)から音声がはじまる。 \\\hline
      ページ数 & PDF & 保存時のページが開く。 \\\hline
    \end{tabular}
  \end{center}
\end{table}
% textlint-enable

% textlint-disable
\begin{table}[htbp]
  \begin{center}
    \caption{本研究において復元しない画面要素}
    \label{tb:intro-not-restore-definition}
    \begin{tabular}{|c|l|l|}
      \hline
      復元しない項目 & 理由 \\\hline\hline
      ピンチズームスケール & read-onlyであるため\cite{pinch-zoom-scale-read-only} \\\hline
      ピクセル比 & read-onlyであるため\cite{pixel-ratio-read-only} \\\hline
    \end{tabular}
  \end{center}
\end{table}
% textlint-enable

\section{システムの特徴}
本節では、Web-Snapshotの特徴的な機能を3つ挙げる。

\subsection{スクロール位置保存・復元機能}
Web-Snapshotは、ブックマーク時のコンテンツのスクロール位置を保存・復元する。
本システムでは、ブックマークの復元時に端末の画面サイズ・ズーム率・使用ブラウザなどの情報をもとにスクロール位置を再計算する。
そのため、異なるブラウザ・端末間でもWebコンテンツの閲覧状態を正しく復元できる。

\subsection{動画/音声の再生位置保存・復元機能}
Web-Snapshotは、ブックマークしたWebコンテンツ内に含まれる動画および音声の再生位置を保存・復元する。
ただし、再生位置を復元できるのは、WebページのDOM内で最上位に位置するもののみである。

\subsection{PDFページ数保存・復元機能}
Web-Snapshotは、ブックマークしたPDFのページ数を保存・復元する。
同一PC内であれば、ローカル上のPDFの保存・復元も可能である。
ただし、本システムではPDFそのものを保存するのではなく、PDFのパスを保存している。
そのため、ローカルPC内でPDFのファイルパスを変更した場合、復元できなくなる。

なお、iOSアプリケーションではPDFページ数の保存機能は提供しない。
その理由は、iOSアプリケーションからブラウザの画面キャプチャを取得する方法が存在しないためである。
% textlint-disable
ただし、Chrome拡張機能で保存したPDFのページ数をiOSアプリケーション内で復元することは可能である。
% textlint-enable
% ここは復元できる。だと意味的におかしいのでチェック外す

\section{システムの使用方法}
本節では、Web-Snapshotシステムの使用方法を、iOSアプリケーション版、Chrome拡張機能版のそれぞれについて説明する。

\subsection{iOSアプリケーション}
本アプリケーションをインストールしたiOS端末で、Safariの「共有ボタン」(図\ref{fig:usage-ios-share})を押す。
すると、図\ref{fig:usage-ios-share-icon}のようにWeb-Snapshotのアイコンが表示される。
このアイコンをクリックすると、図\ref{fig:usage-ios-popup}のように確認用ポップアップが表示される。
ここで"保存する"ボタンを押すことで、Webページをブックマークできる。

本アプリケーションを開くと、図\ref{fig:usage-ios-top}のようなトップ画面が表示される。
ユーザはここでブックマークしたWebページを一覧できる。
その他にも、ブックマークをお気に入りに登録したり、フォルダを作成して整理する機能も提供している。

ブックマークを再度閲覧するには、閲覧したいブックマークをクリックする。
すると、アプリケーション内のブラウザに遷移し、保存したWebページが開く。
Webページがロードされると、スクロール位置・動画再生位置・PDFのページ数などの情報をもとに保存時の閲覧状態を復元する。

\subsection{Chrome拡張機能}
本アプリケーションをChromeにインストールすると、アドレスバーの右にアイコンが表示されるようになる。

アイコンをクリックすると、図\ref{fig:usage-chrome-popup}のようなポップアップが表示される。
ポップアップには、保存済みのコンテンツが5つまで表示される。
その上下に、"保存する"ボタンと"保存済みのコンテンツ一覧"という2つのボタンがある。

"保存する"ボタンを押すことで、Webページをブックマークできる。



"保存済みのコンテンツ一覧"ボタンを押すと、コンテンツ一覧画面に遷移する。
この画面では、ブックマークの管理・お気に入り登録・フォルダ整理などを行うことができる。コンテンツ一覧画面を図\ref{fig:usage-chrome-list}に示す。



ブックマークを再度閲覧するには、閲覧したいブックマークをクリックする。すると、ブラウザ上で新しいタブが開き、保存したWebページを閲覧することが可能になる。
Webページがロードされると、スクロール位置・動画再生位置・PDFのページ数などの情報をもとにブックマーク時の閲覧状態を復元する。

\section{まとめ}
本章では、コンテンツの閲覧状態を保存できるブックマークシステム、Web-Snapshotを提案した。
また、Web-Snapshotシステムの特徴および使用方法を述べた。次章では、本システムの設計について述べる。

% ↓の画像ファイルだけimg/iosに追加。なぜか030以下に入れると読み取ってくれない
\begin{figure}[htbp]
  \begin{tabular}{cc}
    \begin{minipage}[t]{0.45\hsize}
      \caption{【iOS】共有ボタン}
      \label{fig:usage-ios-share}
      \begin{center}
        \includegraphics[bb=0 0 585 1266,width=5cm]{img/ios/usage-ios-share.pdf}
      \end{center}
    \end{minipage} &

    \begin{minipage}[t]{0.45\hsize}
      \caption{【iOS】アイコン}
      \label{fig:usage-ios-share-icon}
      \begin{center}
        \includegraphics[bb=0 0 585 1266,width=5cm]{img/ios/usage-ios-share-icon.pdf}
      \end{center}
    \end{minipage} \\
  
    \begin{minipage}[t]{0.45\hsize}
      \caption{【iOS】確認用ポップアップ}
      \label{fig:usage-ios-popup}
      \begin{center}
        \includegraphics[bb=0 0 585 1266,width=5cm]{img/ios/usage-ios-popup.pdf}
      \end{center}
    \end{minipage} &

    \begin{minipage}[t]{0.45\hsize}
      \caption{【iOS】トップ画面}
      \label{fig:usage-ios-top}
      \begin{center}
        \includegraphics[bb=0 0 585 1266,width=5cm]{img/ios/usage-ios-top.pdf}
      \end{center}
    \end{minipage}
  \end{tabular}
\end{figure}

\begin{figure}[htbp]
  \begin{minipage}[t]{\hsize}
    \caption{【Chrome拡張機能】ポップアップ画面}
    \label{fig:usage-chrome-popup}
    \begin{center}
      \includegraphics[bb=0 0 1280 800,width=15cm]{img/030_web_snapshot_system/chrome/usage-chrome-popup.pdf}
    \end{center}
  \end{minipage} \\

  \begin{minipage}[t]{\hsize}
    \caption{【Chrome拡張機能】コンテンツ一覧画面}
    \label{fig:usage-chrome-list}
    \begin{center}
      \includegraphics[bb=0 0 640 400,width=15cm]{img/030_web_snapshot_system/chrome/usage-chrome-list.pdf}
    \end{center}
  \end{minipage}
\end{figure}