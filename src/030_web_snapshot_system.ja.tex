\chapter{Web-Snapshot システム}
\label{chap:web_snapshot_system}
本章では、コンテンツの閲覧状態を保存できるブックマークシステム、Web-Snapshotを提案する。
はじめにWeb-Snapshotシステムの概要を述べ、次にWeb-Snapshotの特徴を説明する。
そして最後に、ユーザがWeb-Snapshotを利用する流れについて述べる。

\section{システムの概要}
Web-Snapshotはコンテンツの閲覧状態を保存できるブックマークアプリケーションである。
Webコンテンツをブックマークすると、URLとともにスクロール位置・動画再生位置・PDFのページ数などの情報を保存する。
保存されたブックマークはiOSアプリケーションおよびChrome拡張機能から閲覧・管理できる。
ブックマークしたコンテンツを再度開くと、保存時のスクロール位置・動画再生位置・PDFのページ数を復元する。

\section{システムの特徴}
本節では、Web-Snapshotの特徴としてあげられる機能を3つ挙げる。

\subsection{スクロール位置保存・復旧機能}
Web-Snapshotは、ブックマーク時のコンテンツのスクロール位置を保存・復元する。
ブックマークの復元時にデバイスの画面サイズ・ズーム率・使用ブラウザ等の情報をもとにスクロール位置を再計算するため、異なるブラウザ間やデバイス間でもWebコンテンツの閲覧状態を正しく復元できる。

\subsection{動画の再生位置保存・復旧機能}
Web-Snapshotは、ブックマークしたWebコンテンツ内に含まれる動画の再生位置を保存・復元する。
ただし、動画再生位置を復元できるのは、WebコンテンツのDOM内で一番はじめに位置する動画についてのみである。

\subsection{PDFページ数保存・復旧機能}
Web-Snapshotは、ブックマークしたPDFのページ数を保存・復元する。

\subsection{テキストへのリンク作成・保存機能}
\subsection{共有機能}

\section{システムの使用方法}
本節では、Web-Snapshotシステムの使用方法を、iOSアプリケーション版、Chrome拡張機能版のそれぞれについて説明する。

\subsection{iOSアプリケーション}
本アプリケーションをインストールしたiOS端末において、Safariの「共有ボタン」(図\ref{fig:usage-ios-share})を押すと、図\ref{fig:usage-ios-share-icon}のように本アプリケーションのアイコンが表示される。
ブックマークしたいWebページを開いた状態でこのアイコンをクリックすると、図\ref{fig:usage-ios-popup}のように確認用ポップアップが表示される。
ここで"保存する"ボタンを押すことで、Webページをブックマークできる。

本アプリケーションを開くと、図\ref{fig:usage-ios-top}のようなトップ画面が表示される。
ユーザはここで保存したブックマークを一覧できる。
その他にも、ブックマークをお気に入りに登録したり、フォルダを作成して整理する機能も提供している。

ブックマークを再度閲覧するには、閲覧したいブックマークをクリックする。すると、アプリケーション内のブラウザに遷移し、保存したWebページを閲覧できる。
Webページがロードされると、スクロール位置・動画再生位置・PDFのページ数などの情報をもとにブックマーク時の閲覧状態を復元する。

\subsection{Chrome拡張機能}
本アプリケーションをChromeにインストールすると、アドレスバーの右に本アプリケーションのアイコンが表示される。

アイコンをクリックすると、図\ref{fig:usage-chrome-popup}のようなポップアップが表示される。ポップアップには、"保存する"ボタンと"保存済みのコンテンツ一覧"という2つのボタンに加えて、保存済みのコンテンツが5つまで表示される。
"保存する"ボタンを押すことで、Webページをブックマークできる。



"保存済みのコンテンツ一覧"ボタンを押すと、コンテンツ一覧画面に遷移する。
この画面では、ブックマークの管理・お気に入り登録・フォルダ整理などを行うことができる。コンテンツ一覧画面を図\ref{fig:usage-chrome-list}に示す。



ブックマークを再度閲覧するには、閲覧したいブックマークをクリックする。すると、ブラウザ上で新しいタブが開き、保存したWebページを閲覧することが可能になる。
Webページがロードされると、スクロール位置・動画再生位置・PDFのページ数などの情報をもとにブックマーク時の閲覧状態を復元する。

\section{まとめ}
本章では、コンテンツの閲覧状態を保存できるブックマークシステム、Web-Snapshotを提案した。
また、Web-Snapshotシステムの特徴および使用方法を述べた。次章では、本システムの設計について述べる。

\begin{figure}[htbp]
  \begin{tabular}{cc}
    \begin{minipage}[t]{0.45\hsize}
      \caption{【iOS】共有ボタン}
      \label{fig:usage-ios-share}
      \begin{center}
        \includegraphics[bb=0 0 585 1266,width=5cm]{img/030_web_snapshot_system/ios/usage-ios-share.pdf}
      \end{center}
    \end{minipage} &

    \begin{minipage}[t]{0.45\hsize}
      \caption{【iOS】アイコン}
      \label{fig:usage-ios-share-icon}
      \begin{center}
        \includegraphics[bb=0 0 585 1266,width=5cm]{img/030_web_snapshot_system/ios/usage-ios-share-icon.pdf}
      \end{center}
    \end{minipage} \\
  
    \begin{minipage}[t]{0.45\hsize}
      \caption{【iOS】確認用ポップアップ}
      \label{fig:usage-ios-popup}
      \begin{center}
        \includegraphics[bb=0 0 585 1266,width=5cm]{img/030_web_snapshot_system/ios/usage-ios-popup.pdf}
      \end{center}
    \end{minipage} &

    \begin{minipage}[t]{0.45\hsize}
      \caption{【iOS】トップ画面}
      \label{fig:usage-ios-top}
      \begin{center}
        \includegraphics[bb=0 0 585 1266,width=5cm]{img/030_web_snapshot_system/ios/usage-ios-top.pdf}
      \end{center}
    \end{minipage}
  \end{tabular}
\end{figure}

\begin{figure}[htbp]
  \begin{minipage}[t]{\hsize}
    \caption{【Chrome拡張機能】ポップアップ画面}
    \label{fig:usage-chrome-popup}
    \begin{center}
      \includegraphics[bb=0 0 1280 800,width=15cm]{img/030_web_snapshot_system/chrome/usage-chrome-popup.pdf}
    \end{center}
  \end{minipage} \\

  \begin{minipage}[t]{\hsize}
    \caption{【Chrome拡張機能】コンテンツ一覧画面}
    \label{fig:usage-chrome-list}
    \begin{center}
      \includegraphics[bb=0 0 640 400,width=15cm]{img/030_web_snapshot_system/chrome/usage-chrome-list.pdf}
    \end{center}
  \end{minipage}
\end{figure}