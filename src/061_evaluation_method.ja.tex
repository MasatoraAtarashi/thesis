\section{評価環境}
本節では、本研究の評価を行う環境について説明する。

本研究の評価では、Webページの保存・復元を評価するブラウザとしてmacOS Big Sur(バージョン11.6.1)上のChromeを使用する。
同じく、iOS(バージョン15.2.1)上のSafariおよび本アプリケーション内のブラウザを使用する。

\section{評価手法}
本研究では、Webページの閲覧状態の復元の定義を表\ref{tb:re-intro-restore-definition}のように定義した。

% textlint-disable
\begin{table}[htbp]
  \begin{center}
    \caption{【再掲】本研究における閲覧状態の復元の定義}
    \label{tb:re-intro-restore-definition}
    \begin{tabular}{|c|l|l|l|}
      \hline
      復元する項目 & 条件 &  復元の定義 \\\hline\hline
      スクロール位置 & 全て & 
        \begin{tabular}{l}
          保存時の画面左上に位置しているコンテンツが\\画面左上に表示されている。 
        \end{tabular}\\\hline
      動画再生位置 & 動画 & 保存時の再生位置(秒単位)から動画がはじまる。 \\\hline
      音声再生位置 & 音声 & 保存時の再生位置(秒単位)から音声がはじまる。 \\\hline
      ページ数 & PDF & 保存時のページが開く。 \\\hline
    \end{tabular}
  \end{center}
\end{table}
% textlint-enable

今回の評価実験では、実際にWebページをブックマーク・再開し、本研究における復元の定義を満たしているか目視で確認し、復元成功率を示す。
スクロール位置の復元・動画の再生位置の復元・音声の再生位置の復元・PDFのページ数の復元のそれぞれについて、評価に使用するWebページおよび評価方法を説明する。

\subsection{スクロール位置の復元の評価}
本項では、本研究で定義したスクロール位置の復元の評価方法について説明する。

\subsubsection{評価に使用するWebページ}
スクロール位置の復元の評価には、はてなブックマーク上でブックマークされたWebページのうち、2022年01月20日時点のホットエントリ\footnote{ブックマークしたユーザが多いWebページ}のうち重複を省いた28個のサイトを使用する。
上記のサイトを評価に使用する理由は、Web上に存在するページには、一般的にブックマークの対象となるものとならないものがある。
後者の例として、オンライン地図システムやメールシステムなどがある。
本アプリケーションはブックマークシステムであるため、評価には一般的にブックマークの対象となるページに絞る必要がある。
ソーシャルブックマーク上で多数ブックマークされたWebページを使用すれば、その要件を満たせると考え、上記のWebページを評価に使用している。

なお、評価に使用したWebページのURLは付録A\ref{chap:appendix-a}にすべて記載する。

\subsubsection{評価方法}
次に、評価方法について説明する。
スクロール位置の復元の評価では、実際にWebページをブックマーク・再開し、スクロール位置が復元することを目視で確認する。
そして、復元の成功率を算出する。
『スクロール位置の復元』の定義は、第3章\ref{chap:web-snapshot-system-restore-definition}で説明した通りである。

なお、本実験の前提条件として、保存時のピンチズームスケールやピクセル比はデフォルトのままとする。
広告ブロッカーなどの拡張機能は無効の状態を仮定する。

\subsubsection{実験する条件}
本アプリケーションは、複数の端末やブラウザから利用することを想定している。
そのため、本評価では、様々な条件下で正しくスクロール位置が復元することを実験する。
具体的には、まず同じ端末上で保存時と再開時のウィンドウサイズ・ブラウザを揃えて実験する。
Chrome拡張機能版とiOSアプリケーション版のそれぞれについて評価する。

次に、ウィンドウサイズを変えて実験する。
ウィンドウサイズの横幅が保存時と再開時で異なる場合でも、スクロール位置が正しく復元するか検証する。
なお、ウィンドウサイズの縦幅についてはスクロール位置の復元に影響を及ぼさないことが自明であるため実験しない。
また、画面の向きが異なる場合については、本実験で同時に評価可能である。
なぜなら、画面の向きが異なるということは、すなわちウィンドウサイズが異なるということであるためである。

さらに、異なる端末間での復元について評価する。
本システムはPC上のChromeおよびiOS端末上のSafariのみに対応している。
そのため、PC上のChromeで保存した後、iOS端末上のSafariで復元するパターンと、iOS端末上のSafariで保存した後PC上のChromeで復元するパターンの2パターンを実験する。

本実験で用いる条件と実験パターンを表\ref{tb:evl-scroll-position}にまとめる。

% textlint-disable
\begin{table}[htbp]
  \label{tb:evl-scroll-position}
  \caption{実験する条件}
  \begin{center}
    \begin{tabular}{|l|l|}
    \hline
    テスト条件  \\\hline\hline
    保存時と再開時の条件が同じ(Chrome拡張機能) \\ \hline
    保存時と再開時の条件が同じ(iOSアプリケーション) \\ \hline
    再開時のウィンドウサイズが保存時より横に長い  \\ \hline
    再開時のウィンドウサイズが保存時より横に短い  \\ \hline
    保存時はiOSのSafari/再開時はPCのChrome \\ \hline
    保存時はPCのChrome/再開時はiOSのSafari \\ \hline
    \end{tabular}
  \end{center}
\end{table}
% textlint-enable

\subsection{動画の再生位置の復元}
本項では、本研究で定義した『動画の閲覧状態の復元』の評価方法について説明する。

まず、動画の再生位置の復元の評価に使用するWebページについて述べる。
動画の再生位置の復元の評価には、2021年のWebサイト別アクセス数ランキング\cite{The-50-Most-Visited-Websites-in-the-World}などの3つのサイト\cite{mmd-video-research}\cite{popular-video-service}を参考に、特に利用者が多く広く普及していると思われる動画サイト33個をピックアップし、使用する。
評価に用いる動画サービスの一覧を表\ref{tb:evl-video-service-list}に示す。
% なお、評価に使用した動画のURLは付録A\ref{chap:appendix-a}にすべて記載する。

% textlint-disable
\begin{table}[htbp]
  \label{tb:evl-video-service-list}
  \caption{評価に用いる動画サービス一覧}
  \begin{center}
    \begin{tabular}{|l|}
    \hline
    \multicolumn{1}{|c|}{\textbf{動画サービス}} \\\hline
      \begin{tabular}{l}
        Amazon Prime Video, Netflix, hulu, ABEMA, Paravi, TELASA, FODプレミアム, YouTube, \\
        XVideos, Pornhub, xnxx, Zoomレコーディング, xhamster, Twitch, bilibili, Tiktok, \\
        ニコニコ動画, FC2動画, Dailymotion, Gyao!, ツイキャス, パンドラTV, PeeVee, DMM, rakutenTV, \\
        TVer, Mirrativ, Mildom, OPENREC.tv, YOUKU, Sina Video, iQIYI, Tencent Video
      \end{tabular}\\\hline
    \end{tabular}
  \end{center}
\end{table}
% textlint-enable

次に、評価方法について説明する。
動画の再生位置の復元の評価では、実際にWebページ上の動画をブックマーク・再開し、動画の再生位置の復元の成功率を評価する。
『動画再生位置の復元』の定義は、第3章\ref{chap:web-snapshot-system-restore-definition}で説明した通りである。
なお、本実験の前提条件として、広告ブロッカーなどの拡張機能は無効の状態を仮定する。

動画の再生位置の復元の評価において、実験する条件を説明する。
本実験では、表\ref{tb:evl-video-audio-conditions}のように、Chrome拡張機能/iOSアプリケーションでの保存・復元を4パターン検証する。

% textlint-disable
\begin{table}[htbp]
  \label{tb:evl-video-audio-conditions}
  \caption{動画再生位置の復元の評価において実験する条件}
  \begin{center}
    \begin{tabular}{|l|}
    \hline
    \multicolumn{1}{|c|}{\textbf{実験条件}} \\\hline
    Chrome拡張機能で保存→Chrome拡張機能で復元 \\ \hline
    Chrome拡張機能で保存→iOSアプリケーションで復元 \\ \hline
    iOSアプリケーションで保存→Chrome拡張機能で復元 \\ \hline
    iOSアプリケーションで保存→iOSアプリケーションで復元 \\ \hline
    \end{tabular}
  \end{center}
\end{table}
% textlint-enable

\subsection{音声の再生位置の復元の評価}
本項では、本研究で定義した『音声の閲覧状態の復元』の評価方法について説明する。

\subsubsection{評価に使用するWebページ}
まず、音声の再生位置の復元の評価に使用するWebページについて述べる。

% textlint-disable
音声の再生位置の復元の評価には、『いちばんやさしい音声配信ビジネスの教本 人気講師が教える新しいメディアの基礎』\cite{easiest-audio-buisiness-book}に掲載されている音声配信サービスのうち、Webブラウザ上で閲覧できる24個のサービスを使用した。
% textlint-enable
評価に用いる音声配信サービスの一覧を表\ref{tb:evl-audio-service-list}に示す。
% なお、評価に使用した音声のURLは付録A\ref{chap:appendix-a}にすべて記載する。

% textlint-disable
\begin{table}[htbp]
  \label{tb:evl-audio-service-list}
  \caption{評価に用いる音声配信サービス一覧}
  \begin{center}
    \begin{tabular}{|l|}
    \hline
    音声配信サービス  \\\hline\hline
    Audible \\ \hline
    Amazon Music \\ \hline
    LINE MUSIC \\ \hline
    YouTube Music \\ \hline
    うたパス \\ \hline
    Audiostart \\ \hline
    NHKラジオ らじる★らじる \\ \hline
    popln Wave \\ \hline
    himalaya \\ \hline
    Radiotalk \\ \hline
    REC. \\ \hline
    Spoon \\ \hline
    Voicy \\ \hline
    Writone \\ \hline
    こえのブログ \\ \hline
    AuDee \\ \hline
    radiko \\ \hline
    SPINEAR \\ \hline
    SoundCloud \\ \hline
    Castbox \\ \hline
    Google Podcasts \\ \hline
    Podcast Addict \\ \hline
    Spotify for Podcasters \\ \hline
    Wondery \\ \hline
    \end{tabular}
  \end{center}
\end{table}
% textlint-enable

\subsubsection{評価方法}
次に、評価方法について説明する。
音声の再生位置の復元の評価では、実際にWebページ上の音声をブックマーク・再開し、音声の再生位置の復元の成功率を評価する。
『音声再生位置の復元』の定義は、第3章\ref{chap:web-snapshot-system-restore-definition}で説明した通りである。

なお、本実験の前提条件として、広告ブロッカーなどの拡張機能は無効の状態を仮定する。

\subsubsection{実験する条件}
動画の再生位置の復元の評価において、実験する条件を説明する。
本実験では、表\ref{tb:evl-audio-audio-conditions}のように、Chrome拡張機能/iOSアプリケーションでの保存・復元を4パターン検証する。

% textlint-disable
\begin{table}[htbp]
  \label{tb:evl-audio-audio-conditions}
  \caption{音声再生位置の復元の評価において実験する条件}
  \begin{center}
    \begin{tabular}{|l|}
    \hline
    実験条件  \\\hline\hline
    Chrome拡張機能で保存→Chrome拡張機能で復元 \\ \hline
    Chrome拡張機能で保存→iOSアプリケーションで復元 \\ \hline
    iOSアプリケーションで保存→Chrome拡張機能で復元 \\ \hline
    iOSアプリケーションで保存→iOSアプリケーションで復元 \\ \hline
    \end{tabular}
  \end{center}
\end{table}
% textlint-enable

\subsection{PDFのページ数の復元}
本項では、本研究で定義した『PDFの閲覧状態の復元』の評価方法について説明する。

まず、PDFの閲覧状態の復元の評価に使用するPDFについて述べる。
PDFの評価には、PDFを表\ref{tb:evl-pdf-list}の8種類に分け、各種類ごとに5つづつPDF(計40個のPDF)を用意する。
% なお、評価に使用したPDFのURLは付録A\ref{chap:appendix-a}にすべて記載する。

% textlint-disable
\begin{table}[htbp]
  \label{tb:evl-pdf-list}
  \caption{実験に用いるPDFの種類}
  \begin{center}
    \begin{tabular}{|l|}
    \hline
    \multicolumn{1}{|c|}{\textbf{種類}} \\\hline
      \begin{tabular}{l}
        論文, 書籍, 漫画, 説明書, 契約書, IR資料, 会社説明資料, スライド
      \end{tabular}\\\hline
    \end{tabular}
  \end{center}
\end{table}
% textlint-enable

次に、評価方法について説明する。
まず、各PDFからランダムに10ページづつ抽出する。
なお、最大ページ数が10に満たないPDFについては、そのPDFの全てのページでテストする。
そして、各ページを本アプリケーションで保存・再開し、閲覧状態の復元の成功率を評価する。
なお、本実験の前提条件として、広告ブロッカーなどの拡張機能は無効の状態を仮定する。

PDFの閲覧状態の復元の評価において、実験する条件を説明する。
PDFの保存機能はChrome拡張機能でのみサポートしている。
そのため、本実験ではChrome拡張機能での保存および復元のみを実験する。



PDFの評価には、PDFを表\ref{}の8種類に分け、各種類ごとに5つづつPDFを用意する。
その上で、それぞれのPDFについてランダムに10ページを抽出する。
そして、各ページを本アプリケーションで保存・再開し、保存時のページでPDFが開くことを評価する。