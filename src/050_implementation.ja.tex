\chapter{実装}
\label{chap:implementation}
本章では、Web-Snapshotシステムの実装について述べる。
はじめに実装環境について述べ、ついでクライアント側実装、サーバ側実装について説明する。

\section{実装環境}
"バックエンド(WebAPI): Ruby on Rails, Ruby, Heroku

バックエンド(OCR): AWS(Lambda, CloudFormation, S3, Textract), Go

クライアント(iOSアプリ): Swift, XCode

クライアント(Chrome拡張機能): TypeScript"

\section{クライアント側実装}
\subsection{iOSアプリケーション設計}
\subsubsection{ブックマーク保存モジュール}
\subsubsection{ブックマーク一覧モジュール}
\subsubsection{ブックマーク閲覧状態復元モジュール}

\subsection{Chrome拡張機能設計}
\subsubsection{ブックマーク保存モジュール}
\paragraph{スクロール位置保存機能}
\paragraph{動画再生位置保存機能}
\paragraph{PDFページ数保存機能}
\paragraph{テキストフラグメントへのリンク作成・保存機能}

\subsubsection{ブックマーク一覧モジュール}
\paragraph{ブックマーク一覧機能}
\paragraph{フォルダ・お気に入り機能}

\subsubsection{ブックマーク閲覧状態復元モジュール}
\paragraph{スクロール位置復元機能}
\paragraph{動画再生位置復元機能}
\paragraph{PDFページ数復元機能}

\section{サーバ側設計}
\subsection{WebAPI}
\subsection{PDFからページ数を抽出する機能}

\section{まとめ}
本章では、Web-Snapshotシステムの実装について述べた。
次章では、本システムを利用しいて実際にWebページをブックマークし、閲覧状態の復元を評価する。