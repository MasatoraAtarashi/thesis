\begin{jabstract}

  現在では一人のユーザが複数の端末を所有して使い分けることが当たり前になっている。
  ユーザはWebコンテンツを複数の端末・ブラウザから閲覧するために、ブックマークシステムを利用することがある。
  しかし、現在のブックマークシステムでは、WebページをURL単位でしか保存・復元できない。
  そのため、ページを保存した際にユーザがどこまでコンテンツを閲覧していたかという情報が失われてしまう。
  その結果、ユーザはWebページを訪れる度に、コンテンツ内の目当ての文言・シーンを見つけ直さなければならない。  

  そこで本研究では、ユーザがブックマークしたWebページを再度訪れた際に、保存時の閲覧状態を復元できるブックマークシステムを提案する。
  ユーザがWebコンテンツをブックマークすると、URLとともにスクロール位置・動画再生位置・PDFのページ数などの情報を保存する。
  ユーザがブックマークしたWebページを再度訪れた際に、保存時のスクロール位置・動画再生位置・PDFのページ数などを設定することで、閲覧状態を復元する。

  本研究では、ブックマークシステムのクライアントとしてiOSアプリケーション・Chrome拡張機能をそれぞれSwift・TypeScriptで開発した。
  加えて、ブックマークデータを保存するためのデータベースサーバとWebAPIをRubyで実装した。
  さらに、PDFからページ数を抽出するモジュールをGolangで実装した。

  閲覧状態を復元できることを評価するため、実装したブックマークシステムを用いて様々な種類のWebページをブックマークした。
  そして、複数の端末やブラウザでブックマークしたWebページを再度閲覧した。

  (評価実施後詳細を書く。)
  その結果、〇〇ということが分かった。

  (結論の要約をする。)

\end{jabstract}
