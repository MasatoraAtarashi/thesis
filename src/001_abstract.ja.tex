\begin{jabstract}

  現在では、多くの人がWebブラウザを利用して様々なWebサイト・動画・PDFなどのコンテンツを閲覧している。
  ユーザのWebブラウジング体験を改善するために、ブックマークという手法が普及している。

  しかし、Webコンテンツ内の特定のリソースを指し示す方法が多数存在するにも関わらず、既存のブックマークシステムでWebページ単位でしか保存・復元を行えない。
  その結果、ユーザはWebページを訪れる度に、コンテンツ内の目当ての文言・シーンを見つけ直さなければならない。

  本研究では、前述の課題を解決するために、WebページのURLとともにブックマーク時の閲覧状態を保存・復元できるブックマークシステムWeb-Snapshotを提案する。
  Web-Snapshotシステムは、ブックマーク時にスクロール位置・動画再生位置・PDFのページ数などのデータを取得する。
  ユーザが再度Webページを訪れた際に、それらのデータに基づいて閲覧状態を復元する。

  ※評価した後書く。

  開発したWeb-Snapshotシステムを用いて実際にWebページをブックマークした結果、ということがわかった。
  有効性が示唆された。

\end{jabstract}
