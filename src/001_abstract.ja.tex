\begin{jabstract}

ハードウェアの小型化・高性能化により、いろいろなものがインターネットに接続できるようになったことで、IoTが広がっている。

しかし、ハードウェアが多様化する一方、それらのソフトウェア開発は各プラットフォームごとにC/C++を書くのが一般的である。

そこで、ソフトウェア開発を効率化するため、単一のコードを様々なプラットフォームで動かせるようにしたい。

また、ソフトウェアアップデート(動的な実行内容の変更)も実現したい。

本研究では、IoT端末で動作するWebAssembly実行環境を実装し、既存の実装手法(ネイティブ、JVM)とフットプリントやランタイム・バイナリのサイズを比較する。

また、動的に実行内容を変更できることを示す。

また、WebAssemblyで記述したコード同士で通信できることを示す。

\end{jabstract}
