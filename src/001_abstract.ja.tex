\begin{jabstract}

現在、マイコンにおけるプログラムは静的に実行されるのが主流である。
すなわち、プログラムはフラッシュメモリ等に書き込まれ、起動時にブートローダがプログラムを参照して実行する。
そのため、処理内容を変更するためには、有線接続によって、もしくは無線接続によるOTA(Over-the-Air)アップデートによって、フラッシュメモリ上のプログラムを書き換えた上で再起動する必要がある。

一方、Webページとして提供されるWebアプリケーションでは、プログラムは動的に実行される。
Webブラウザはアプリケーション・プラットフォームとして、Webページに記述されたスクリプトや外部へのリンクを辿り、必要なプログラムを取得した上で実行する。
これにより、アクセスするWebページごとに異なるアプリケーションを提供することができる上に、利用者の環境や状態によって処理内容を変更することもできる。

本論文では、動的なアプリケーションの実行モデルをマイコン上でも実現するため、WebAssemblyを実行形式として用いたアプリケーション・プラットフォームの設計を提案する。
WebAssemblyは、Webブラウザ上で実行するためのプログラムを記述できる仮想的な命令セットアーキテクチャであり、Webブラウザにおける動的なアプリケーション実行モデルを高速化するために設計された。
Webブラウザでの動作を第一義的な目的として設計されたWebAssemblyだが、仕様としてWebブラウザ内での動作に限定する制約はない。

本研究では、マイコンにおいてWebAssemblyプログラムの動的な取得と実行を行う、マイコン向けWebAssembly実行環境を設計した。
その上で、本実行環境の実現可能性を検討するため、C言語でWebAssemblyインタプリタを実装した。
実装した実行環境を用いてESP32およびIntel Core i5 3GHzのCPUを搭載したPCで同一のWebAssemblyプログラムをそれぞれ実行した。その結果、同一のWebAssemblyプログラムを実行するために、ESP32ではPCの7倍から15倍程度のクロック数を必要とすることが分かった。
この結果を、同じWebAssemblyプログラムをPCのWebブラウザ上で実行した際の実行性能と比較した結果、30番目のフィボナッチ数を560ミリ秒で計算できる性能となることが推測された。

\end{jabstract}
