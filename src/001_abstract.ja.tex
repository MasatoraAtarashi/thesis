\begin{jabstract}

組み込み向けデバイスの性能向上と低価格化により、身の回りの家具や家電をはじめとしたあらゆるものが
インターネットに接続し、協調して動作することで生活を豊かにするInternet of Things(IoT)が
普及しつつある。

Espressif Systemsによって開発されたESP32は、そうした組み込み向けデバイスの中でも
比較的高性能かつ廉価なデバイスの一つである。ESP32はデュアルコアCPUと500KB以上のRAMを搭載し、
自身のみでWi-FiやBluetooth、Bluetooth Low Energyによる接続が行える。

ESP32を用いることでインターネットに接続するハードウェアを低コストで実装することができるが、
ESP32上で動作するソフトウェアの開発には、ESP32が搭載するCPUの命令セットアーキテクチャである
Xtensaアーキテクチャをサポートする限られた数の開発環境しか用いることができない。
このことは、開発者がソフトウェア開発に用いることのできるプログラミング言語やツールの選択肢を
狭めている。

そこで、本研究では、ESP32向けソフトウェアの開発に汎用的な開発環境を使用できるようにすることを目的として、
命令セットアーキテクチャとしてのサポートが広がりつつあるWebAssemblyの実行環境を実装した。
また、汎用的なコンパイラで生成したWebAssemblyプログラムをESP32上で実行できることを確認した。
さらに、ある同一の内容のプログラムの実行における実行時間およびメモリフットプリントを、
本実行環境を用いる場合と用いない場合とで比較した。

その結果、実行時間において約3倍〜10倍、メモリフットプリントにおいて約2倍〜5倍のオーバーヘッドがあることがわかったものの、
ESP32において実用的なオーバーヘッドでWebAssemblyを用いることが可能であることを示した。
これにより、ESP32上でWebAssembly実行環境を実装することで汎用的な開発環境を用いることが可能になることを示すことができた。

\end{jabstract}
