\begin{jabstract}

組み込み向けデバイスの性能向上と低価格化により、身の回りの家具や家電をはじめとしたあらゆるものが
インターネットに接続し、協調して動作することで生活を豊かにするInternet of Things(IoT)が
普及しつつある。

Espressif Systemsによって開発されたESP32\cite{esp32}は、そうした組み込み向けデバイスの中でも
比較的高性能かつ廉価なデバイスの一つである。ESP32はデュアルコアCPUと500KB以上のRAMを搭載し、
自身のみでWi-FiやBluetooth、Bluetooth Low Energyによる接続が行える。

ESP32を用いることでインターネットに接続するハードウェアを低コストで実装することができる一方で、
ESP32上で動作するソフトウェアの開発環境は、ESP32が搭載するXtensa命令セットアーキテクチャを
サポートするものでなくてはならない。
このことは、開発者がソフトウェア開発に用いることのできるプログラミング言語やツールの選択肢を
狭めている。

そこで、本研究では、幅広い開発環境でサポートされている命令セットアーキテクチャをESP32上で動作させる
ことを目的として、仮想命令セットアーキテクチャであるWebAssemblyの実行環境を実装した。
また、同一の内容のプログラムについて、実装した実行環境における実行速度およびメモリフットプリントを
ネイティブ実行と比較した。その結果、実行速度において約3倍〜10倍、メモリフットプリントにおいて
約2倍〜5倍のオーバーヘッドがあることがわかった。この結果から、
ESP32における仮想命令セットアーキテクチャとして、実用的なオーバーヘッドでWebAssemblyを用いることが
可能であることを示した。

\end{jabstract}
