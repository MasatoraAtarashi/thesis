\subsection{動画の再生位置の復元}
本項では、本研究で定義した『動画の閲覧状態の復元』機能について実験する。

\subsubsection{評価に使用するWebページ}
まず、動画の再生位置の復元の評価に使用するWebページについて述べる。
動画の再生位置の復元の評価には、2021年のWebサイト別アクセス数ランキング\cite{The-50-Most-Visited-Websites-in-the-World}などの3つのサイト\cite{mmd-video-research}\cite{popular-video-service}を参考に、特に利用者が多く広く普及していると思われる動画サイト33個をピックアップし、使用する。
なお、評価に使用した動画サイトおよび動画のURLは付録A\ref{chap:appendix-a}にすべて記載する。

\subsubsection{評価方法}
次に、評価方法について説明する。
動画の再生位置の復元の評価では、実際にWebページ上の動画をブックマーク・再開し、動画の再生位置が復元することを確認する。
『動画再生位置の復元』の定義は、第3章\ref{chap:web-snapshot-system-restore-definition}で説明した通りである。

なお、本実験の前提条件として、広告ブロッカーなどの拡張機能は無効の状態を仮定する。

\subsubsection{実験する条件}
動画の再生位置の復元の評価において、実験する条件を説明する。
本実験では、表\ref{tb:evl-video-audio-conditions}のように、Chrome拡張機能/iOSアプリケーションでの保存・復元を4パターン検証する。

% textlint-disable
\begin{table}[htbp]
  \label{tb:evl-video-audio-conditions}
  \caption{動画再生位置の復元の評価において実験する条件}
  \begin{center}
    \begin{tabular}{|l|}
    \hline
    実験条件  \\ \hline
    Chrome拡張機能で保存→Chrome拡張機能で復元 \\ \hline
    Chrome拡張機能で保存→iOSアプリケーションで復元 \\ \hline
    iOSアプリケーションで保存→Chrome拡張機能で復元 \\ \hline
    iOSアプリケーションで保存→iOSアプリケーションで復元 \\ \hline
    \end{tabular}
  \end{center}
\end{table}
% textlint-enable
