\subsection{動画/音声の復元}
本項では、本研究で定義した『動画/音声の閲覧状態の復元』機能について実験する。
具体的には、基本的な復元の定義を満たした上で、保存時の再生位置から動画/音声が再生されることを確認する。
評価実験では、表\ref{tb:evl-video-audio-conditions}のように、Chrome拡張機能/iOSアプリケーションでの保存・復元を4パターン検証する。

% textlint-disable
\begin{table}[htbp]
  \label{tb:evl-video-audio-conditions}
  \caption{実験する条件}
  \begin{center}
    \begin{tabular}{|l|}
    \hline
    実験条件  \\ \hline
    Chrome拡張機能で保存→Chrome拡張機能で復元 \\ \hline
    Chrome拡張機能で保存→iOSアプリケーションで復元 \\ \hline
    iOSアプリケーションで保存→Chrome拡張機能で復元 \\ \hline
    iOSアプリケーションで保存→iOSアプリケーションで復元 \\ \hline
    \end{tabular}
  \end{center}
\end{table}
% textlint-enable

\subsubsection{動画の復元}
動画の閲覧状態を復元できることを確認するために、利用者数の多い動画配信サービスで検証する。
Webブラウザが対応している主な動画形式についてそれぞれ検証する。
検証する動画サービスと、実験に用いる動画のURLを表\ref{tb:evl-video-check-list}に示す。

% textlint-disable
\begin{table}[htbp]
  \begin{center}
    \caption{検証する動画サービス}
    \label{tb:evl-video-check-list}
    \begin{tabular}{|c|l|l|}
      \hline
      動画サービス & URL \\\hline\hline
      YouTube & 検証に使うWebページのURLを記入予定 \\\hline
      Amazon Prime Video & 検証に使うWebページのURLを記入予定 \\\hline
      Netflix & 検証に使うWebページのURLを記入予定 \\\hline
      ABEMA & 検証に使うWebページのURLを記入予定 \\\hline
    \end{tabular}
  \end{center}
\end{table}
% textlint-enable