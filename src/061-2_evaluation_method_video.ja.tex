\subsection{動画の再生位置の復元}
本項では、本研究で定義した『動画の閲覧状態の復元』の評価方法について説明する。

まず、動画の再生位置の復元の評価に使用するWebページについて述べる。
動画の再生位置の復元の評価には、2021年のWebサイト別アクセス数ランキング\cite{The-50-Most-Visited-Websites-in-the-World}などの3つのサイト\cite{mmd-video-research}\cite{popular-video-service}を参考に、特に利用者が多く広く普及していると思われる動画サイト33個をピックアップし、使用する。
評価に用いる動画サービスの一覧を表\ref{tb:evl-video-service-list}に示す。
% なお、評価に使用した動画のURLは付録A\ref{chap:appendix-a}にすべて記載する。

% textlint-disable
\begin{table}[htbp]
  \label{tb:evl-video-service-list}
  \caption{評価に用いる動画サービス一覧}
  \begin{center}
    \begin{tabular}{|l|}
    \hline
    \multicolumn{1}{|c|}{\textbf{動画サービス}} \\\hline
      \begin{tabular}{l}
        Amazon Prime Video, Netflix, hulu, ABEMA, Paravi, TELASA, FODプレミアム, YouTube, \\
        XVideos, Pornhub, xnxx, Zoomレコーディング, xhamster, Twitch, bilibili, Tiktok, \\
        ニコニコ動画, FC2動画, Dailymotion, Gyao!, ツイキャス, パンドラTV, PeeVee, DMM, rakutenTV, \\
        TVer, Mirrativ, Mildom, OPENREC.tv, YOUKU, Sina Video, iQIYI, Tencent Video
      \end{tabular}\\\hline
    \end{tabular}
  \end{center}
\end{table}
% textlint-enable

次に、評価方法について説明する。
動画の再生位置の復元の評価では、実際にWebページ上の動画をブックマーク・再開し、動画の再生位置の復元の成功率を評価する。
『動画再生位置の復元』の定義は、第3章\ref{chap:web-snapshot-system-restore-definition}で説明した通りである。
なお、本実験の前提条件として、広告ブロッカーなどの拡張機能は無効の状態を仮定する。

動画の再生位置の復元の評価において、実験する条件を説明する。
本実験では、表\ref{tb:evl-video-audio-conditions}のように、Chrome拡張機能/iOSアプリケーションでの保存・復元を4パターン検証する。

% textlint-disable
\begin{table}[htbp]
  \label{tb:evl-video-audio-conditions}
  \caption{動画再生位置の復元の評価において実験する条件}
  \begin{center}
    \begin{tabular}{|l|}
    \hline
    \multicolumn{1}{|c|}{\textbf{実験条件}} \\\hline
    Chrome拡張機能で保存→Chrome拡張機能で復元 \\ \hline
    Chrome拡張機能で保存→iOSアプリケーションで復元 \\ \hline
    iOSアプリケーションで保存→Chrome拡張機能で復元 \\ \hline
    iOSアプリケーションで保存→iOSアプリケーションで復元 \\ \hline
    \end{tabular}
  \end{center}
\end{table}
% textlint-enable
