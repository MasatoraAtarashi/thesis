

\section{クライアント側実装}
クライアントはiOSアプリケーションおよびChrome拡張機能であり、それぞれ主にSwift、TypeScriptによって実装した。
それぞれのアプリケーションについて、認証モジュール・ブックマーク保存モジュール・ブックマーク一覧モジュール・ブックマークの閲覧状態を復元するモジュールの4つのモジュールについて説明する。

\subsection{iOSアプリケーション実装}
本節では、iOSアプリケーションの実装について述べる。
iOSアプリケーションでは、基本的な復元及び動画・音声の復元の2種類の復元に対応している。

\subsubsection{認証モジュール}
本項では、認証モジュールについて説明する。
ユーザが本アプリケーションを開くと、図\ref{fig:impl-ios-top-not-auth-view}のような画面が表示される。この画面の”登録”ボタンを押すと図\ref{fig:impl-ios-auth-view}が開き、会員登録できる。
会員登録のためには、メールアドレスとパスワードを入力する方法と、外部サービスを通じて登録する方法の2種類から選ぶことができる。

\begin{figure}[htbp]
  \begin{tabular}{cc}
    \begin{minipage}[t]{0.45\hsize}
      \caption{【iOS】初期画面}
      \label{fig:impl-ios-top-not-auth-view}
      \begin{center}
        \includegraphics[bb=0 0 585 1266,width=5cm]{img/050_implementation/ios/ios-top-not-auth-view.pdf}
      \end{center}
    \end{minipage} &

    \begin{minipage}[t]{0.45\hsize}
      \caption{【iOS】認証画面}
      \label{fig:impl-ios-auth-view}
      \begin{center}
        \includegraphics[bb=0 0 585 1266,width=5cm]{img/050_implementation/ios/ios-auth-view.pdf}
      \end{center}
    \end{minipage}
  \end{tabular}
\end{figure}

メールアドレスで登録するには、図\ref{fig:impl-ios-auth-view}でメールアドレスとパスワードを入力する。
ユーザが必要な情報を入力した上で”登録”ボタンを押すと、AuthenticationViewControllerがサーバにデータを送信する。
会員登録が完了すると、サーバは認証に用いるトークンをヘッダに含めてレスポンスを返す。
AuthenticationViewControllerでは、取得したアクセストークンをKeyChain\cite{}に保存する。
ヘッダにこのアクセストークンを付与してリクエストを送信することで、サーバが提供するAPIを利用できるようになる。

外部サービスを通じてログインする場合は、図\ref{fig:impl-ios-auth-view}の下部にあるボタンをクリックする。
例として、”Twitterで続ける”ボタンを押すと、Twitter\cite{}の認証画面が開く。
本画面でTwitterアカウントにログインし、本アプリケーションとの連携を許可すると、OAuthの仕組みを利用して本アプリケーションに登録できる。

一度会員登録すると、以後アプリケーションを利用する際は自動でログインされる。
なお、iOSアプリケーションは会員登録せずとも利用できる。その場合、ブックマークのデータは端末内のストレージに保存される。

\subsubsection{ブックマーク保存モジュール}
本項ではブックマーク保存モジュールについて説明する。
ブックマーク保存モジュールはユーザがブックマークしたいWebページのデータと閲覧状態を取得し、サーバに保存するモジュールである。
iOSアプリケーションでは、ブックマークの保存は標準のブラウザであるSafariから行う。
ユーザがブックマークしたいWebページを開いた状態で図\ref{fig:usage-ios-share}のような共有ボタンを押すと、共有機能に対応したアプリケーションのアイコンが並んで表示される。
この中から本アプリケーションのアイコンをクリックすると、図\ref{fig:usage-ios-popup}のような確認用ポップアップが表示される。
ここで”保存”ボタンを押すと、Webコンテンツのデータの取得およびサーバへの保存が実行される。

上記の機能の実装にはAppleが提供しているShareExtension\cite{}を利用している。
ShareExtentionとは、Appleが提供するAppExtension\cite{app-extension}の1つで、任意のアプリケーション内のデータを開発者が提供するアプリケーションに共有できるAPIである。
ユーザが"保存"ボタンを押したタイミングで、ブラウザ上でJavaScriptコードを実行し、Webページのデータと閲覧状態を表すデータを取得する。
表\ref{tb:impl-ios-data-js-api}に、JavaScriptで取得するデータと取得するために用いるAPIをまとめる。
なお、動画の再生位置はWebページ内のDOMにおいて一番初めに位置する動画のデータを取得する。Webページ内に動画が存在しない場合、ゼロ値は0である。

% textlint-disable
\begin{table}[htbp]
  \label{tb:impl-ios-data-js-api}
  \caption{ブックマーク保存モジュールで取得するデータ}
  \begin{center}
    \begin{tabular}{|l|l|l|}
    \hline
    取得するデータ & 型 & API \\ \hline
    URL & String & Document.URL \\ \hline
    ページタイトル & String & Document.title \\ \hline
    スクロール位置(縦) & Int & Element.scrollTop \\ \hline
    スクロール位置(横) & Int & Element.scrollLeft \\ \hline
    ページ長(縦) & Int & Element.scrollHeight \\ \hline
    ページ長(横) & Int & Element.scrollWidth \\ \hline
    動画再生位置 & Double & HTMLMediaElement.currentTime \\ \hline
    ズーム率 & Int &  \\ \hline
    \end{tabular}
  \end{center}
\end{table}
% textlint-enable

ShareExtensionを通じて取得したデータは、iOSアプリケーション内のShareViewControllerというコントローラーに渡される。
ShareViewControllerは、取得したデータにブラウザの情報をメタデータとして付け加え、サーバに送信する。

図\ref{}にiOSアプリケーションにおけるブックマーク保存モジュールの全体像を示す。

\subsubsection{一覧モジュール}
本項ではブックマーク一覧モジュールについて説明する。
一覧モジュールでは、前項のブックマーク保存モジュールで保存したブックマークを一覧できる。
ユーザが本アプリケーションを開くと、サーバからユーザの保存したブックマークを取得する。
取得したデータからタイトル・URL・保存日時・サムネイル画像を抽出し、図\ref{fig:usage-ios-top}のようにカード型で一覧表示する。

ユーザがブックマークをクリックすると、閲覧状態復元モジュールに遷移し、ブックマークしたWebページを開いた上で閲覧状態を復元する。
ブックマークの削除やお気に入り登録、フォルダ分け、検索機能など、ブックマークアプリケーションで一般的に存在する機能も本モジュールで提供している。

\subsubsection{閲覧状態復元モジュール}
\label{sec:impl-client-ios-restore-module}

本項では閲覧状態復元モジュールについて説明する。
前項のブックマーク一覧モジュールで、ユーザが再度閲覧したいブックマークをクリックすると、閲覧状態復元モジュールが開く。

閲覧状態復元モジュールの機能の実装はWebViewControllerというコントローラーが担う。
WebViewControllerは、WKWebViewというiOSアプリ内専用のブラウザでブックマークしたWebページを表示する。
ブックマークが選択されると、WebViewControllerは保存時と現在のスクロール位置/ページ長/ブラウザ/ズーム率などの情報から適切なスクロール位置を計算する。
この計算結果に基づいてWebページをスクロールすると、ブックマーク時点で画面上限に位置していたコンテンツが、WKWebViewの画面の上限にちょうど表示される。
このスクロール位置を計算するアルゴリズムは以下の通りである。

\begin{equation}
スクロール位置 = スクロール位置
\end{equation}

Webページの読み込みが完了すると、WebViewController内のwebView(\_:didFinish:)\cite{didFinish}というハンドラが呼び出される。
webView(\_:didFinish:)ハンドラは、閲覧状態の復元を司る関数で、スクロール位置と動画再生位置を設定する。
まず、webView(\_:didFinish:)ハンドラは、window.scrollToというAPIを利用して計算結果の分だけWebページをスクロールする。
その上で、Webページ内に動画が存在する場合は、videoタグのcurrentTimeに保存時の動画再生位置を設定する。
なお、保存時と同様、再生位置を復元するのはWebページ内のDOMの最上位の動画のみである。

