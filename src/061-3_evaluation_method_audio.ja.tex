\subsection{音声の再生位置の復元}
本項では、本研究で定義した『音声の閲覧状態の復元』の評価方法について説明する。

まず、音声の再生位置の復元の評価に使用するWebページについて述べる。

% textlint-disable
音声の再生位置の復元の評価には、『いちばんやさしい音声配信ビジネスの教本 人気講師が教える新しいメディアの基礎』\cite{easiest-audio-buisiness-book}に掲載されている音声配信サービスのうち、Webブラウザ上で閲覧できる24個のサービスを使用した。
% textlint-enable
評価に用いる音声配信サービスの一覧を表\ref{tb:evl-audio-service-list}に示す。
% なお、評価に使用した音声のURLは付録A\ref{chap:appendix-a}にすべて記載する。

% textlint-disable
\begin{table}[htbp]
  \label{tb:evl-audio-service-list}
  \caption{評価に用いる音声配信サービス一覧}
  \begin{center}
    \begin{tabular}{|l|}
    \hline
    \multicolumn{1}{|c|}{\textbf{音声配信サービス}} \\\hline
    Audible \\ \hline
    Amazon Music \\ \hline
    LINE MUSIC \\ \hline
    YouTube Music \\ \hline
    うたパス \\ \hline
    Audiostart \\ \hline
    NHKラジオ らじる★らじる \\ \hline
    popln Wave \\ \hline
    himalaya \\ \hline
    Radiotalk \\ \hline
    REC. \\ \hline
    Spoon \\ \hline
    Voicy \\ \hline
    Writone \\ \hline
    こえのブログ \\ \hline
    AuDee \\ \hline
    radiko \\ \hline
    SPINEAR \\ \hline
    SoundCloud \\ \hline
    Castbox \\ \hline
    Google Podcasts \\ \hline
    Podcast Addict \\ \hline
    Spotify for Podcasters \\ \hline
    Wondery \\ \hline
    \end{tabular}
  \end{center}
\end{table}
% textlint-enable

次に、評価方法について説明する。
音声の再生位置の復元の評価では、実際にWebページ上の音声をブックマーク・再開し、音声の再生位置の復元の成功率を評価する。
『音声再生位置の復元』の定義は、第3章\ref{chap:web-snapshot-system-restore-definition}で説明した通りである。
なお、本実験の前提条件として、広告ブロッカーなどの拡張機能は無効の状態を仮定する。

動画の再生位置の復元の評価において、実験する条件を説明する。
本実験では、表\ref{tb:evl-audio-audio-conditions}のように、Chrome拡張機能/iOSアプリケーションでの保存・復元を4パターン検証する。

% textlint-disable
\begin{table}[htbp]
  \label{tb:evl-audio-audio-conditions}
  \caption{音声再生位置の復元の評価において実験する条件}
  \begin{center}
    \begin{tabular}{|l|}
    \hline
    \multicolumn{1}{|c|}{\textbf{実験条件}} \\\hline
    Chrome拡張機能で保存→Chrome拡張機能で復元 \\ \hline
    Chrome拡張機能で保存→iOSアプリケーションで復元 \\ \hline
    iOSアプリケーションで保存→Chrome拡張機能で復元 \\ \hline
    iOSアプリケーションで保存→iOSアプリケーションで復元 \\ \hline
    \end{tabular}
  \end{center}
\end{table}
% textlint-enable
