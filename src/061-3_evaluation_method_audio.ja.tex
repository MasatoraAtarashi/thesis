\subsection{音声の再生位置の復元}
本項では、本研究で定義した『音声の閲覧状態の復元』機能について実験する。

\subsubsection{評価に使用するWebページ}
まず、音声の再生位置の復元の評価に使用するWebページについて述べる。
音声の再生位置の復元の評価には、『いちばんやさしい音声配信ビジネスの教本 人気講師が教える新しいメディアの基礎』\cite{}に掲載されている音声配信サービスのうち、Webブラウザ上で閲覧できる24個のサービスを使用した。
なお、評価に使用した音声配信サービスおよび音声のURLは付録A\ref{chap:appendix-a}にすべて記載する。

\subsubsection{評価方法}
次に、評価方法について説明する。
音声の再生位置の復元の評価では、実際にWebページ上の音声をブックマーク・再開し、音声の再生位置が復元することを確認する。
『音声再生位置の復元』の定義は、第3章\ref{chap:web-snapshot-system-restore-definition}で説明した通りである。

なお、本実験の前提条件として、広告ブロッカーなどの拡張機能は無効の状態を仮定する。

\subsubsection{実験する条件}
動画の再生位置の復元の評価において、実験する条件を説明する。
本実験では、表\ref{tb:evl-audio-audio-conditions}のように、Chrome拡張機能/iOSアプリケーションでの保存・復元を4パターン検証する。

% textlint-disable
\begin{table}[htbp]
  \label{tb:evl-audio-audio-conditions}
  \caption{音声再生位置の復元の評価において実験する条件}
  \begin{center}
    \begin{tabular}{|l|}
    \hline
    実験条件  \\ \hline
    Chrome拡張機能で保存→Chrome拡張機能で復元 \\ \hline
    Chrome拡張機能で保存→iOSアプリケーションで復元 \\ \hline
    iOSアプリケーションで保存→Chrome拡張機能で復元 \\ \hline
    iOSアプリケーションで保存→iOSアプリケーションで復元 \\ \hline
    \end{tabular}
  \end{center}
\end{table}
% textlint-enable
