\chapter{結論}
\label{chap:conclusion}
本章では、本論文のまとめと今後の課題について述べる。

\section{本研究のまとめ}
\label{section:conclusion}
本研究では、保存時のWebブラウザ上の閲覧状態が失われてしまうという既存のブックマークシステムの問題に着目した。
それを解決するため、Webブラウザ上のスクロール位置や、動画・音声・PDFをどこまで見ていたか、といったような覧状態を保存・復元できるブックマークシステムWeb Snapshotを提案・開発した。
Web SnapshotシステムはクライアントとしてiOSアプリケーションおよびChrome拡張機能を持ち、バックエンドにWebAPIを持つアプリケーションになっている。

実装したWeb Snapshotシステムを用いて、実際にWebページ・動画サイト・音声配信サービス・PDFをブックマークし、再開時に保存時の閲覧状態が復元することを評価した。
その結果、スクロール位置は平均して約53\%の確率で復元に成功した。
動画の再生位置は平均して約47\%の確率で復元に成功した。
音声の再生位置は平均して約21\%の確率で復元に成功した。
PDFのページ数は平均して約70\%の確率で復元に成功した。

以上の結果から、本システムを利用することで、一定の条件下であればスクロール位置・動画の再生位置・PDFのページ数といったようなWebコンテンツの閲覧状態を復元できることがわかった。
一方で、端末間を跨いだ復元ではスクロール位置・動画・音声のいずれも復元に失敗するケースが多く、本システムの課題であることがわかった。
また、音声の再生位置の復元は多くのサービスで失敗しており、改善が必要である。

\section{今後の課題と展望}
本節では、本研究で提案したWeb Snapshotシステムの今後の課題と展望を述べる。

\subsection{スクロール位置の復元}
本研究では、同一の端末間であれば高い確率でスクロール位置を復元できたものの、端末を跨いだ復元はほとんど実現できなかった。
本システムではスクロール位置を復元するためにまず保存時のブラウザのウィンドウサイズを復元する。
しかし、PCからスマートフォンのように、保存時と復元時のブラウザのウィンドウサイズがあまりにも違う場合、ウィンドウサイズの復元ができないためである。
そこで、今後の展望として、ブラウザのウィンドウサイズを変えるのではなくHTMLの本文表示領域の幅を制御する手法を検討したい。
ただし、上記の手法はいずれもユーザの利便性を損ねる方法でもあるため、慎重に考慮したい。

また、いくつかのWebページでは広告がページ読み込み完了後に差し込まれるため、スクロール位置がずれてしまっていた。
この問題への対策として、広告ブロッカーのような機能の実装を検討している。

加えて、本研究ではブラウザのズーム率を考慮できていない。
今後はブラウザのズーム率も復元することで、より完全なスクロール位置の復元を実現したい。

\subsection{動画再生位置の復元}
本研究では、Chrome拡張機能での動画再生位置の復元は約70\%の確率で成功している一方で、iOSアプリでは成功率が低くなっている。
これはアプリケーション内ブラウザとして採用しているWKWebViewの不具合によるものと推測される。
しかし、アプリケーション内ブラウザはAppleが実装しており、オープンソースではないためデバッグは困難であるが、フォーラムに投稿するなど改善を呼びかけたい。

また、いくつかの動画サイトでは動画広告の再生位置を誤って取得してしまっていた。
これは、そもそも本システムでHTML上の最上位に位置する動画の再生位置のみを取得・復元する仕様であることが原因である。
今後はWebページ内の全ての動画の再生位置を復元できるような機能を実装したい。

加えて、一部のサイトではHTMLのvideoタグを使用せずに動画ビューワを実装していた。
本システムでは、videoタグの使用を前提としているため、調査した上で改善したい。

ただし、上記を全て実現したとしても、一部の動画サイトでは外部からのJavaScriptの実行をブロックしているものもあり、全てのサイトで閲覧状態の復元を実現できるわけではない。

\subsection{音声再生位置の復元}
本研究では、音声の再生位置の復元はほとんどのサービスで失敗している。
これは本システムの手法はHTMLのaudioタグを使用して音声を再生していることを前提としているが、実際にはaudioタグを使用せずに実装しているサービスがほとんどであったためである。
そこで、多くの音声配信サービスが採用している音声再生機能の実装について調査し、対応したい。

また、iOSアプリケーションでは動画の再生位置の復元と同様に復元が機能しない例もいくつかあった。
動画の再生位置の復元と同様に、この点についても改善を促したい。
HTMLの最上位に位置する音声のみの復元に留まっている点も、動画と同様である。
この点についても改善を検討したい。

\subsection{PDFのページ数の復元}
PDFのページ数の復元機能は、現状Chrome拡張機能上でしか実現できていない。
これには2つの理由がある。
1つ目は、iOSアプリケーションからSafariの。
2つ目は、iOSのブラウザではURLのフラグメントによるPDFの表示ページ数の制御ができないためである。
前者については、本システムをiOSアプリケーションとしてではなく、Safari向けの拡張機能として提供すれば実現できるかも知れない。
後者については、Safariやアプリケーション内ブラウザが対応することを待ちたい。

また、本システムではページ内に絵や図などが大きく表示されているPDFや、非ビジネス用途のPDFでは復元成功率が低かった。
この問題点への対処方法として、まずPDFの画面キャプチャを撮影時にPDFのコンテンツを表示している部分を切り取る処理を入れることを検討している。
独自のOCR用学習済みモデルを構築することも目指したい。