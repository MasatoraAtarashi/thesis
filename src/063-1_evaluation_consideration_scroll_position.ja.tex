\subsection{スクロール位置の復元}

\subsubsection{復元に失敗した原因の考察}
スクロール位置の復元は、実験条件によって成功率に大きな差が見られた。
保存・再開をともにChrome拡張機能で行った場合の成功率は、保存時・復元時のブラウザのウィンドウサイズに関わらず約86\%だった。
一方、保存をiOSのSafariで、再開をiOSアプリケーションで行った場合の成功率は約54\%と比較的低い結果となった。
端末・ブラウザを跨いだ復元はほとんど成功しなかった。

\paragraph{保存・再開をともにChrome拡張機能で行ったケース}

保存・再開をともにChrome拡張機能で行ったケースでは、復元に失敗したWebページは2種類に分けられる。

1種類目は、広告がページの読み込み完了後に差し込まれるWebページで、3つのWebページが該当する。
本アプリケーションでは、ページの読み込みが完了したタイミングでスクロール位置を復元しようとする。
その結果、このようなWebページではスクロール後にさらにコンテンツが変化してしまい、スクロール位置を正しく復元できない。

2種類目は、本アプリケーションによるスクロールが機能しないWebページで、1つのWebページが該当する。
これは、WebページがJavaScriptの実行をブロックしているものと推測される。
上記の失敗要因と該当するWebページの割合を図\ref{fig:evl-consideration-scroll-position-cause-ratio-chrome}に示す。

% textlint-disable
\begin{figure}[htbp]
  \label{fig:evl-consideration-scroll-position-cause-ratio-chrome}
  \begin{center}
    \includegraphics[bb=0 0 452.22222222 293.33333333,width=15cm]{img/060_evaluation/consideration/scroll_position/cause-ratio-chrome.pdf}
  \end{center}
  \caption{失敗要因と該当するWebページの割合(保存・再開をともにChrome拡張機能で行ったケース)}
\end{figure}
% textlint-enable

\paragraph{保存をiOSのSafariで、再開をiOSアプリケーションで行ったケース}

保存をiOSのSafariで、再開をiOSアプリケーションで行ったケースでは、復元に失敗したWebページは4種類に分けられる。

1・2種類目は、保存・再開をともにChrome拡張機能で行ったケースにおいて説明したものと同様である。
なお、iOSアプリケーションでは2種類目の原因である本アプリケーションによるスクロールが機能しないWebページは2つ存在した。

3種類目は、iOSのSafariと本アプリケーション内のブラウザとでWebページのヘッダの表示領域が異なるページであり、7つのWebページが該当する。
これは図\ref{}および\ref{}に示すように、Safariとアプリケーション内ブラウザ間で画面の表示領域が異なるために、同じ座標までスクロールしても表示がずれてしまう。

% textlint-disable
\begin{figure}[htbp]
  \begin{tabular}{cc}
    \begin{minipage}[t]{0.45\hsize}
      \label{fig:evl-consideration-scroll-position-ios-diff-safari}
      \begin{center}
        \includegraphics[bb=0 0 585 1266,width=5cm]{img/060_evaluation/consideration/scroll_position/example/ios-diff-safari.pdf}
      \end{center}
      \caption{Safariとアプリ内のブラウザで表示が異なるページの例(Safari)}
    \end{minipage} &

    \begin{minipage}[t]{0.45\hsize}
      \label{fig:evl-consideration-scroll-position-ios-diff-wkwebview}
      \begin{center}
        \includegraphics[bb=0 0 585 1266,width=5cm]{img/060_evaluation/consideration/scroll_position/example/ios-diff-wkwebview.pdf}
      \end{center}
      \caption{Safariとアプリ内のブラウザで表示が異なるページの例(アプリ内ブラウザ)}
    \end{minipage}
  \end{tabular}
\end{figure}
% textlint-enable

4種類目は、広告がコンテンツに重なる形で表示されてしまうWebページである。1つのWebページが該当する。
これは図\ref{fig:evl-consideration-scroll-position-example-ios-ad-overrap}に示すような表示で、スマートフォンのブラウザのみで起きる現象だった。

% textlint-disable
\begin{figure}[htbp]
  \label{fig:evl-consideration-scroll-position-example-ios-ad-overrap}
  \begin{center}
    \includegraphics[bb=0 0 585 1266,width=5cm]{img/060_evaluation/consideration/scroll_position/example/ios-ad-overrap.pdf}
  \end{center}
  \caption{広告がコンテンツに重なるようにして表示されてしまうWebページ}
\end{figure}
% textlint-enable

上記の失敗要因と該当するWebページの割合を図\ref{fig:evl-consideration-scroll-position-cause-ratio-ios}に示す。

% textlint-disable
\begin{figure}[htbp]
  \label{fig:evl-consideration-scroll-position-cause-ratio-ios}
  \begin{center}
    \includegraphics[bb=0 0 626.22222222 378.88888889,width=15cm]{img/060_evaluation/consideration/scroll_position/cause-ratio-ios.pdf}
  \end{center}
  \caption{失敗要因と該当するWebページの割合(保存をiOSのSafariで、再開をiOSアプリケーションで行ったケース)}
\end{figure}
% textlint-enable

\paragraph{端末・ブラウザを跨いだ復元}

端末・ブラウザを跨いだ復元では、失敗の原因は3種類に分けられる。

1種類目は、PCとスマートフォンで表示を変えているWebページで、10つのWebページが該当する。

2種類目は、PCとスマートフォンでブラウザのウィンドウサイズが大きく異なるために、ウィンドウの幅を復元することができないWebページである。
17つのWebページが該当する。
本アプリケーションでは、保存時のブラウザのウィンドウサイズを復元し、その上で保存時の画面左上の座標まで画面をスクロールすることでスクロール位置を復元している。
しかし、Chromeには最小ウィンドウサイズが存在するため、保存時のウィンドウサイズが再開時に比べて大幅に小さい場合、復元することができない。
また、iOSアプリケーションではアプリ内ブラウザのサイズを変更できないため、スクロール位置の復元が困難である。

3種類目は、本アプリケーションによるスクロールが機能しないWebページで、1つのWebページが該当する。
上記の失敗要因と該当するWebページの割合を図\ref{fig:evl-consideration-scroll-position-cause-ratio-multi-device}に示す。

% textlint-disable
\begin{figure}[htbp]
  \label{fig:evl-consideration-scroll-position-cause-ratio-multi-device}
  \begin{center}
    \includegraphics[bb=0 0 452.22222222 303.33333333,width=15cm]{img/060_evaluation/consideration/scroll_position/cause-ratio-multi-device.pdf}
  \end{center}
  \caption{失敗要因と該当するWebページの割合(保存時はMacのChrome・復元時はiOSのsafariで実験したケース)}
\end{figure}
% textlint-enable

\subsubsection{改善点についての考察}
これらの結果から、スクロール位置の復元成功率を向上させるためには、3つの方法が考えられる。

1つ目は、アドブロッカーなどを有効にすることで広告を非表示にする方法である。
広告が非表示であれば、広告のサイズが可変であったり、ページの読み込み完了後に広告が差し込まれるケースを防止することができる。

2つ目は、ブックマークからWebページを開く際にUserAgentを偽装する方法である。
再開時にブックマーク時のUserAgentでHTTPリクエストを送信することで、保存時と全く同じWebコンテンツを取得する。
それにより、ブラウザを跨いだスクロール位置の復元を実現できる可能性がある。

3つ目は、ブックマークからWebページを開く際に、ブラウザのウィンドウサイズではなくHTMLの表示領域の幅を調整する方法である。
ブラウザのウィンドウサイズは最小サイズが決まっており、保存時と再開時でブラウザサイズが大きく変わる場合に対応できない。
だが、HTMLの表示領域には最小サイズが存在しないため、上記のケースにも対応できる可能性がある。
ただし、この方法を取る場合はWebページ内の文字のサイズも同時に復元する必要があり、これは比較的難易度が高いと思われる。
なぜなら、文字のサイズはブラウザが自動で制御しているため、適切な文字サイズを計算するロジックが不明なためである。

上記の点を改善したとしても、なお課題は残っている。

1点目の課題は、本アプリケーションではWebページのURLと閲覧状態を表すメタデータを保存しているだけであるため、ページの内容が書き換わってしまった場合に復元ができなくなってしまう点である。
これは本アプリケーションに限らず、ブックマークシステム全てが抱えている課題でもある。

2点目の課題は、上記の方法ではブックマークからWebページを開いた際に、自動的にブラウザのウィンドウサイズが変更されたり、画面の表示サイズが変わってしまう点である。
そのため、利用者のブラウジング体験を大きく損なう可能性がある。
利用者の利便性と復元の厳密さはトレードオフの関係にあるため、どこまで厳密にスクロール位置を復元するかという点は検討の余地がある。