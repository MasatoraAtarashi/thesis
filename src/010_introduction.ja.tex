\chapter{序論}
\label{chap:introduction}

本章では,はじめに本研究における背景を述べる。
ついで,本研究の目的およびアプローチを述べる。
最後に本論文の構成を示す。

\section{背景}
\label{section:background}

現在では、多くの人がWebブラウザを利用して様々なWebサイト・動画・PDFなどのコンテンツを利用している。
標準化団体やブラウザー業界団体、Webサービス事業者などが、ユーザのWebブラウジング体験を改善するために様々な手法を提案している。

その一例として、ブックマークという手法がある。
任意のWebサイトのURIを登録(ブックマーク)しておくと、1クリックでそのWebサイトを開くことができるようになる、という機能である。
ユーザが逐一Webブラウザで検索したり、URLを入力する必要がなくなる、というメリットがある。
史上初のグラフィカルなWebブラウザであるNCSA Mosaic\cite{ncsa-mosaic}に導入されて以来、機能をほとんど変えずに普及してきた。
現在では、Webブラウザ(Chrome\cite{chrome}/Safari\cite{safari}/Firefox\cite{firefox})がそれぞれ機能を提供している他、Pocket\cite{pocket}/Instapaper\cite{instapaper}などブックマーク機能を専用で提供するWebサービスも多数存在する。

その他にも、Webコンテンツ内の様々なリソースを指し示すためのインターフェイスを提供することで、ユーザの利便性を向上させるという手法も存在する。
例えば、URIのフラグメント\footnote{URLの末尾に\#をつけて指定することでページ内のリソースを指し示す機能}にHTML内のリソースのidを指定することで、Webサイト内の特定の項目を指し示すことができる(例: \url{https://example.com#some-id})。
あるいは、動画共有サイトのYouTube\cite{youtube}では、\url{https://youtu.be/some-video?t=500}といったようにクエリパラメータで再生位置を指定して動画を開く機能を提供している。
また、WebブラウザのPDFビューワでは、ページ数を指定してPDFを開けるような仕様になっている場合も多い\cite{browser-pdf-viewer-pdf-page-num-function}(例: \url{https://example.pdf#page=5})。

\section{本研究が着目する課題}
本研究では、Webコンテンツ内の特定のリソースを指し示す方法が多数存在するにも関わらず、ブックマークシステムでWebページ単位でしか保存・復元を行えない点に着目する。
現在のブックマークシステムでは、ページを保存した際にユーザがどこまでコンテンツを閲覧していたかという情報が失われてしまう。
そのため、ユーザはWebページを訪れる度に、コンテンツ内の目当ての文言・シーンを見つけ直さなければならない。

また、現在では一人のユーザが複数の端末を所有して使い分けることが当たり前になっている。
ユーザはWebコンテンツを複数の端末・ブラウザから閲覧するために、ブックマークシステムを利用することがある。
しかし、現在のブックマークシステムでは、複数の端末間でWebコンテンツの閲覧状態を同期させることができない。
そのため、ユーザはデバイスを切り替えるたびに目当ての情報を探し直す必要がある。

Webページ内の様々なリソースを指定してブックマークできれば、上記のような問題を解決できる。
結果として、Webブラウジングにおけるユーザの体験を大きく向上させることができる。

\section{本研究の目的とアプローチ}
本研究では、前述の課題を解決するために、WebページのURIとともにスクロール位置・動画再生位置・PDFのページ数などの閲覧状態を保存・復元できるブックマークシステムを開発する。

本研究における『復元』の定義は、"保存時の画面上限に位置しているコンテンツが画面上限に表示されていること"とする。
動画および音声については、上記に加えて保存時と同じ再生位置から再生が始まることを条件とする。
ただし、動画や音声は、WebページのDOM\footnote{Document Object Model}内で最上位に位置するもののみ、復元する。
例外として、PDFについては保存時と同じページ数が開くことを復元と定義する。

本研究における復元の定義を以下の表\ref{tb:intro-restore-definition}にまとめる。

\begin{table}[htbp]
  \begin{center}
    \caption{本研究における閲覧状態の復元の定義}
    \label{tb:intro-restore-definition}
    \begin{tabular}{|c|l|l|}
      \hline
      Webコンテンツの種類 & 復元の定義 \\\hline\hline
      基本 & 保存時の画面上限に位置しているコンテンツが画面上限に表示されている。 \\\hline
      動画・音声 & 保存時の再生位置から動画がはじまる。 \\\hline
      PDF & 保存時のページが開く。 \\\hline
    \end{tabular}
  \end{center}
\end{table}

(コメント: 復元の定義のその他の候補)

\begin{itemize}
  \item 保存時の画面内に表示されているコンテンツがそのまま表示される
  \item 保存時の画面中央に表示されているコンテンツが画面中央に表示されている
  \item 保存時の画面左上に位置しているコンテンツが画面左上に表示されている
\end{itemize}

\section{本論文の構成}

本論文における以降の構成は次の通りである。

\ref{chap:related_works}章では、既存のブックマークシステムについて整理し、本研究との差分について述べる。

\ref{chap:web_snapshot_system}章では、本研究で提案するシステムの概要について述べる。
本研究では、Webページとともにスクロール位置・動画再生位置・PDFのページ数などの閲覧状態を保存・復元できるブックマークシステムを提案する。

\ref{chap:design}章では、本研究で提案するシステムの設計について述べる。

\ref{chap:implementation}章では、本研究で実装したブックマークシステムについて述べる。
本研究では、クライアントとしてiOSアプリケーション・Chrome拡張機能、サーバ側にWebAPIおよびPDFのスクリーンショットからページ数を抽出するモジュールを実装した。

\ref{chap:evaluation}章では、本研究で実装したブックマークシステムを用いて実際にWebページを保存し、閲覧状態の復元を評価する。

\ref{chap:conclusion}章では、本研究における結論を示す。