
\subsection{PDFのページ数の復元}

\subsubsection{復元に失敗した原因の考察}
PDFのページ数の復元では、PDFの種類によって復元成功率に大きな差が見られた。
説明書(マニュアル)や契約書、IR資料などでは85\%を超える確率で復元に成功した一方で、漫画・書籍はおよそ半分の確率で復元に失敗している。

PDFのページ数の復元に失敗したケースでは、全て画面キャプチャからOCRでページ数を抽出する処理に失敗している。
これは、今回OCRに利用したAWSのTextractというサービスの2つの特徴が原因であると考えられる。

% textlint-disable
第一に、Textractは
\begin{quote}
Amazon Textract の学習済み機械学習モデルは、請求書、領収書、契約書、税務文書、注文書、登録フォーム、給付申請書、保険金請求書、保険証書など、ほぼすべての業界の数千万におよぶドキュメント
\end{quote}
を学習データとしている、という特徴を持っている\cite{textract}。
% textlint-enable
このために、ビジネス向けのPDFとそうでないPDFでページ数の検出成功率が異なる結果になったのではないかと考えられる。
実際に、今回の評価実験でテストしたPDFをビジネス向け(説明書/契約書/IR資料/会社説明資料)とそうでないもの(論文/書籍/漫画/スライド)に分けた場合の成功率を図\ref{fig:success-rate-business-or-not-pdf}に示す。
ビジネス向けのPDFでは約86\%と高い成功率を示している一方で、非ビジネス向けのPDFでは約56\%しか成功していない。

第二に、Textractは
\begin{quote}
ドキュメントのレイアウトとページ上の重要な要素を自動的に検出
\end{quote}
するという特徴を持っている。
そのため、図\ref{fig:example-figure-pdf}\cite{example-pdf}のようにページ内に絵や図などが大きく表示されている場合、それらを自動的に重要な要素として認識してしまう。
その結果、ナビゲーションバーに表示されているPDFのページ数を認識しづらくなるのではないかと考えられる。
実際に、今回の評価実験で使用したPDFをページ内に絵や図が大きく表示されているものとそうでないものに分けた場合の復元成功率を\ref{fig:success-rate-figure-or-not-pdf}に示す。
ページ内に絵や図が大きく表示されているPDFでは復元成功率は約46\%と低いのに対して、そうでないPDFでは約79\%と高い成功率を示している。

% textlint-disable
\begin{figure}[htbp]
  \label{fig:example-figure-pdf}
  \begin{center}
    \includegraphics[bb=0 0 1918 998,width=15cm]{img/060_evaluation/consideration/pdf/example/example-pdf.pdf}
  \end{center}
  \caption{ページ内に絵や図などが大きく表示されているPDFの例}
\end{figure}

\begin{figure}[htbp]
  \label{fig:success-rate-business-or-not-pdf}
  \begin{center}
    \includegraphics[bb=0 0 600 371,width=15cm]{img/060_evaluation/consideration/pdf/success-rate-business-or-not-pdf.pdf}
  \end{center}
  \caption{ビジネス向けPDFとそうでないPDFの復元成功率}
\end{figure}

\begin{figure}[htbp]
  \label{fig:success-rate-figure-or-not-pdf}
  \begin{center}
    \includegraphics[bb=0 0 600 371,width=15cm]{img/060_evaluation/consideration/pdf/success-rate-figure-or-not-pdf.pdf}
  \end{center}
  \caption{ページ内に絵・図等が大きく表示されているPDFとそうでないPDFの復元成功率}
\end{figure}
% textlint-enable

\subsubsection{改善点についての考察}
これらの結果から、今回の評価で成功率の低かった種類のPDFの復元成功率を向上させるためには、2つの方法が考えられる。

1つ目は、画面キャプチャを保存時に文書の中身が表示されている領域を切り取り、ナビゲーションバーのみの画像をOCRにかける処理を追加する方法である。
今回の実験で、PDFの種類によってページ数抽出処理の成功率は大きく変化することがわかった。
そして、その原因はページ内に絵や図など正確な識字を妨げる要素が存在することであった。
画面キャプチャからPDFの中身を表示している部分を削除できれば、上記の点を解消できるのではないだろうか。

2つ目は、新しいOCRの学習済みモデルを構築することである。
具体的には、表示域内に図や絵が大きく表示されている非ビジネス向けPDFを学習データとして用いて、専用の学習済みモデルを構築すれば、PDFの復元成功率を改善できるのではないか。
