\chapter{結論}
\label{chap:conclusion}

\section{本研究のまとめ}
\label{section:conclusion}

本研究では、マイコンにおける実行内容をWebAssemblyで記述することで動的なプログラムの実行が可能になるという仮説のもと、マイコン向けWebAssembly実行環境を設計した。
また、その実現可能性を評価するため、WebAssemblyバイナリのインタプリタをC言語で実装した。

実装したWebAssemblyインタプリタを用いて、ESP32上およびPC上で同一のWebAsseblyプログラムを実行し、実行時間およびメモリフットプリントを計測した。
実行時間では、ESP32上での実行はPC上での実行に比べ、クロック数による比較でおよそ7倍から15倍の時間がかかることが分かった。
この結果をPC上のWebブラウザの実行性能と比較し、ESP32上で〜なWebAssemblyインタプリタが実現できた場合、30番目のフィボナッチ数の計算を約560ミリ秒で行えることが分かった。
また、メモリフットプリントについて、208バイトのメモリ消費で関数呼び出しが行えることが分かった。

\section{今後の課題}

本研究では、静的なWebAssemblyバイナリの実行を検証した。
ネットワーク上から取得し、また実行終了後にネットワーク上からバイナリを更新するためには、通信のための計算負荷・メモリ消費も想定する必要がある。
また、本研究では100バイト未満のバイナリをマイコン上で実行できることを確認したが、より大きなバイナリを実行する際には、ネットワークから流れてくるバイト列を逐一パースし実行するなどの工夫が必要だろう。

また、本研究で実装した実行環境では、ホストプログラムからWebAssemblyプログラムの関数を一方的に呼び出せるのみであったが、入出力を実現するためにはWebAssemblyプログラムからマイコンの機能へのアクセス手段を提供する必要がある。
