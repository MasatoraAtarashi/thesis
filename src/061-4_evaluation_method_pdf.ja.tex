\subsection{PDFの復元の評価}
本項では、本研究で定義した『PDFの閲覧状態の復元』の評価方法について説明する。


\subsubsection{評価に使用するPDF}
まず、PDFの閲覧状態の復元の評価に使用するPDFについて述べる。
PDFの評価には、PDFを表\ref{}の8種類に分け、各種類ごとに5つづつPDF(計40個のPDF)を用意する。
なお、評価に使用したPDFのURLは付録A\ref{chap:appendix-a}にすべて記載する。

% textlint-disable
\begin{table}[htbp]
  \label{tb:evl-pdf-list}
  \caption{実験に用いるPDFの種類}
  \begin{center}
    \begin{tabular}{|l|}
    \hline
    種類  \\\hline\hline
    論文 \\ \hline
    書籍 \\ \hline
    漫画 \\ \hline
    説明書 \\ \hline
    契約書 \\ \hline
    IR資料 \\ \hline
    会社説明資料 \\ \hline
    スライド \\ \hline
    \end{tabular}
  \end{center}
\end{table}
% textlint-enable

\subsubsection{評価方法}
次に、評価方法について説明する。

まず、各PDFについてランダムに10ページを抽出する。
そして、各ページを本アプリケーションで保存・再開し、閲覧状態の復元の成功率を評価する。
なお、本実験の前提条件として、広告ブロッカーなどの拡張機能は無効の状態を仮定する。

\subsubsection{実験する条件}
PDFの閲覧状態の復元の評価において、実験する条件を説明する。
PDFの保存機能はChrome拡張機能でのみサポートしている。
そのため、本実験ではChrome拡張機能での保存および復元のみを実験する。
