\subsection{PDFの復元}
PDFについては、再開時に保存時のページが開くことを検証する。

% 検証内容について(確率、ローカル、ch→ios, ch→ch)
PDFの保存機能はChrome拡張機能でのみサポートしている。
そのため、本実験ではChrome拡張機能で保存した後、Chrome拡張機能/iOSアプリケーションで再開し、それぞれ評価する。
本機能はOCRを用いて実現している。
PDFの中身やコンディションによってはOCRが失敗する可能性もある。
そのため、評価では1つのPDFにつき10回づつ実験し、成功率を示す。
また、本機能はWeb上にホスティングされているPDFとローカルPC上のPDFの両方に対応している。
ローカルPC上のPDFについては、同一PC内でのみ復元可能であるため、同一PC内での復元のみ検証する。
本評価でテストするパターンを表\ref{tb:evl-pdf-conditions}に示す。

本評価では、実際のユースケースを想定して様々な種類のPDFを実験に用いる。
具体的には、論文や書籍、漫画・説明書・契約書・IR資料・会社説明資料・スライドの復元を検証する。
検証に用いるPDFの一覧を表\ref{tb:evl-pdf-list}に示す。

% textlint-disable
\begin{table}[htbp]
  \label{tb:evl-pdf-conditions}
  \caption{実験する条件}
  \begin{center}
    \begin{tabular}{|l|}
    \hline
    実験条件  \\ \hline
    Chrome拡張機能で保存→Chromeで復元(ローカルPDF) \\ \hline
    Chrome拡張機能で保存→Chromeで復元 \\ \hline
    Chrome拡張機能で保存→iOSアプリケーションで復元 \\ \hline
    \end{tabular}
  \end{center}
\end{table}

\begin{table}[htbp]
  \label{tb:evl-pdf-list}
  \caption{実験に用いるPDF}
  \begin{center}
    \begin{tabular}{|l|l|}
    \hline
    種別 & PDFのリンク  \\ \hline
    論文 & 検証に使うPDFのURLを記入予定  \\ \hline
    書籍 & 検証に使うPDFのURLを記入予定  \\ \hline
    漫画 & 検証に使うPDFのURLを記入予定  \\ \hline
    説明書 & 検証に使うPDFのURLを記入予定  \\ \hline
    契約書 & 検証に使うPDFのURLを記入予定  \\ \hline
    IR資料 & 検証に使うPDFのURLを記入予定  \\ \hline
    会社説明資料 & 検証に使うPDFのURLを記入予定  \\ \hline
    スライド & 検証に使うPDFのURLを記入予定  \\ \hline
    \end{tabular}
  \end{center}
\end{table}
% textlint-enable
