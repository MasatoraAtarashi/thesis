\newif\ifjapanese

\japanesetrue

\ifjapanese
  \documentclass[a4j,11pt]{jreport}
  %%% binder:
  % \documentclass[a4j,twoside,openright,11pt]{jreport}

  \renewcommand{\bibname}{参考文献}
  \newcommand{\acknowledgmentname}{謝辞}
\else
  \documentclass[a4paper,11pt]{report}
  \newcommand{\acknowledgmentname}{Acknowledgment}
\fi

\usepackage{../styles/thesis}
\usepackage{ascmac}
\usepackage{graphicx}
\usepackage{multirow}
\usepackage{url}
\usepackage{otf}
\usepackage[dvipdfmx]{hyperref}
\usepackage{pxjahyper}
\usepackage{listings,jvlisting}
\usepackage{threeparttable}
\bibliographystyle{jplain}

\jclass   {卒業論文}
\jtitle   {Web-Snapshot: 異なるブラウザ・端末間で閲覧状態を保存・再現できるブックマークシステム}
\juniv    {慶應義塾大学}
\jfaculty {環境情報学部}
\jauthor  {新 真虎}
\jryear   {4}
\jsyear   {2022}
\jkeyword {ブックマーク, Web, UX}
\jproject {徳田・村井・楠本・中村・高汐・バンミーター・植原・三次・中澤・武田 合同研究プロジェクト}
\jdate    {2022年1月}

\eclass   {Bachelor's Thesis}
\etitle   {Web-Snapshot: Bookmark system that can save and reproduce browsing status between different browsers and devises}
\euniv    {Keio University}
\efaculty {Bachelor of Arts in Environment and Information Studies}
\eauthor  {Masatora Atarashi}
\eyear    {2022}
\ekeyword {Bookmark, Web, UX}
\eproject {Tokuda/Murai/Kusumoto/Nakamura/Takashio/Van Meter/Uehara/Mitsugi/Nakazawa/Takeda Labs}
\edate    {January 2022}