\subsection{スクロール位置の復元}
本項では、スクロール位置の復元についての実験結果について評価し、考察する。
表\ref{tb:evl-result-scroll-position}に実験結果を示す。

% textlint-disable
\begin{table}[htbp]
  \caption{スクロール位置の復元の実験結果}
  \label{tb:evl-result-scroll-position}
  \begin{center}
    \begin{tabular}{rrr}
      \hline
      実験条件 & スクロール位置(縦)の復元 & スクロール位置(横)の復元 \\ \hline \hline
      再開時のウィンドウサイズが\\
      保存時より縦に長い & 成功or失敗を書きます & 成功or失敗を書きます \\ \hline
      再開時のウィンドウサイズが\\
      保存時より縦に短い & -- & -- \\ \hline
      再開時のウィンドウサイズが\\
      保存時より横に長い & -- & -- \\ \hline
      再開時のウィンドウサイズが\\
      保存時より横に短い & -- & -- \\ \hline
      再開時のウィンドウサイズが\\
      保存時より縦横に長い & -- & -- \\ \hline
      再開時のウィンドウサイズが\\
      保存時より縦横に短い & -- & -- \\ \hline
      保存時は画面の向きが縦/\\
      再開時は画面の向きが横 & -- & -- \\ \hline
      保存時は画面の向きが横/\\
      再開時は画面の向きが縦 & -- & -- \\ \hline
      保存時はモバイル端末(iPhone12)/\\
      再開時はタブレット(iPad) & -- & -- \\ \hline
      保存時はタブレット(iPad)/\\
      再開時はモバイル端末(iPhone12) & -- & -- \\ \hline
      保存時はiOSのSafari/\\
      再開時はPCのChrome & -- & -- \\ \hline
      保存時はPCのChrome/\\
      再開時はiOSのSafari & -- & -- \\ \hline
      保存時と再開時の条件が\\
      同じ(Chrome拡張機能) & -- & -- \\ \hline
      保存時と再開時の条件が\\
      同じ(iOSアプリケーション) & -- & -- \\ \hline
    \end{tabular}
  \end{center}
\end{table}
% textlint-enable