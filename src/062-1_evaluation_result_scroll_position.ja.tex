\subsection{スクロール位置の復元}
本項では、スクロール位置の復元について、本評価実験で得られた評価結果を示す。

まず、保存時と再開時の条件が同じ場合の評価結果を説明する。
Chrome拡張機能(条件1とする)では、評価を行った28個のWebページのうち、全体の約86\%にあたる24個のページでスクロール位置を正しく復元できた。
iOSアプリケーション(条件2とする)では、評価を行った28個のWebページのうち、全体の約54\%にあたる15個のページでのみ、スクロール位置を正しく復元できた。

次に、保存時と再開時のブラウザウィンドウの幅が異なる場合の評価結果を示す。
再開時のウィンドウサイズが保存時より横に長い場合(条件3とする)では、評価を行った28個のWebページのうち、全体の約86\%にあたる24個のページでスクロール位置を正しく復元できた。
再開時のウィンドウサイズが保存時より横に短い場合(条件4とする)では、評価を行った28個のWebページのうち、全体の約86\%にあたる24個のページでスクロール位置を正しく復元できた。

最後に、端末(ブラウザ)を跨いだ評価実験の結果を述べる。
iOSのSafariで保存し、Chrome拡張機能から再開した場合(条件5とする)では、評価を行った28個のWebページのうち、全体の約7\%にあたる2つのページでのみ、スクロール位置を正しく復元できた。
一方、Chrome拡張機能で保存し、iOSアプリケーションから再開した場合(条件6とする)では、評価を行った28個のWebページのうち、全てのサイトで復元できなかった。

上記に示した、条件ごとの復元成功率を図\ref{fig:success-rate-scroll-position}に示す。

% textlint-disable
\begin{figure}[htbp]
  \caption{条件別のスクロール位置の復元成功率}
  \label{fig:success-rate-scroll-position}
  \begin{center}
    \includegraphics[bb=0 0 600 371,width=15cm]{img/060_evaluation/result/scroll_position/success-rate-scroll-position.pdf}
  \end{center}
\end{figure}
% textlint-enable
