\subsection{スクロール位置の復元の評価}
本項では、本研究で定義したスクロール位置の復元の評価方法について説明する。

\subsubsection{評価に使用するWebページ}
スクロール位置の復元の評価には、はてなブックマーク上でブックマークされたWebページのうち、2022年01月20日時点のホットエントリ\footnote{ブックマークしたユーザが多いWebページ}のうち重複を省いた28個のサイトを使用する。
上記のサイトを評価に使用する理由は、Web上に存在するページには、一般的にブックマークの対象となるものとならないものがある。
後者の例として、オンライン地図システムやメールシステムなどがある。
本アプリケーションはブックマークシステムであるため、評価には一般的にブックマークの対象となるページに絞る必要がある。
ソーシャルブックマーク上で多数ブックマークされたWebページを使用すれば、その要件を満たせると考え、上記のWebページを評価に使用している。

なお、評価に使用したWebページのURLは付録A\ref{chap:appendix-a}にすべて記載する。

\subsubsection{評価方法}
次に、評価方法について説明する。
スクロール位置の復元の評価では、実際にWebページをブックマーク・再開し、スクロール位置が復元することを目視で確認する。
そして、復元の成功率を算出する。
『スクロール位置の復元』の定義は、第3章\ref{chap:web-snapshot-system-restore-definition}で説明した通りである。

なお、本実験の前提条件として、保存時のピンチズームスケールやピクセル比はデフォルトのままとする。
広告ブロッカーなどの拡張機能は無効の状態を仮定する。

\subsubsection{実験する条件}
本アプリケーションは、複数の端末やブラウザから利用することを想定している。
そのため、本評価では、様々な条件下で正しくスクロール位置が復元することを実験する。
具体的には、まず同じ端末上で保存時と再開時のウィンドウサイズ・ブラウザを揃えて実験する。
Chrome拡張機能版とiOSアプリケーション版のそれぞれについて評価する。

次に、ウィンドウサイズを変えて実験する。
ウィンドウサイズの横幅が保存時と再開時で異なる場合でも、スクロール位置が正しく復元するか検証する。
本実験は、Chrome拡張機能のブラウザウィンドウの幅を変えることで実験する。
なお、ウィンドウサイズの縦幅についてはスクロール位置の復元に影響を及ぼさないことが自明であるため実験しない。
また、画面の向きが異なる場合については、本実験で同時に評価可能である。
なぜなら、画面の向きが異なるということは、すなわちウィンドウサイズが異なるということであるためである。

さらに、異なる端末間での復元について評価する。
本システムはPC上のChromeおよびiOS端末上のSafariのみに対応している。
そのため、PC上のChromeで保存した後、iOS端末上のSafariで復元するパターンと、iOS端末上のSafariで保存した後PC上のChromeで復元するパターンの2パターンを実験する。

本実験で用いる条件と実験パターンを表\ref{tb:evl-scroll-position}にまとめる。

% textlint-disable
\begin{table}[htbp]
  \label{tb:evl-scroll-position}
  \caption{実験する条件}
  \begin{center}
    \begin{tabular}{|l|l|}
    \hline
    テスト条件  \\\hline\hline
    保存時と再開時の条件が同じ(Chrome拡張機能) \\ \hline
    保存時と再開時の条件が同じ(iOSアプリケーション) \\ \hline
    再開時のウィンドウサイズが保存時より横に長い  \\ \hline
    再開時のウィンドウサイズが保存時より横に短い  \\ \hline
    保存時はiOSのSafari/再開時はPCのChrome \\ \hline
    保存時はPCのChrome/再開時はiOSのSafari \\ \hline
    \end{tabular}
  \end{center}
\end{table}
% textlint-enable
