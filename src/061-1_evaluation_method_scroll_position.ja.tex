\subsection{スクロール位置の復元}
本項では、本研究で定義したスクロール位置の復元の評価方法について説明する。

% 評価項目・手法
まず、評価手法について説明する。
スクロール位置の復元の評価では、縦横のスクロール位置がそれぞれ復元することを目視で確認する。
なお、『スクロール位置の復元』の定義は、第3章\ref{chap:web-snapshot-system-restore-definition}で説明した通りである。

% 実験条件
本評価では、様々な条件下で正しくスクロール位置が復元することを実験する。
具体的には、まず同じ端末上で保存時と再開時のウィンドウサイズ・画面の向き・ブラウザを揃えて実験する。
Chrome拡張機能版とiOSアプリケーション版のそれぞれについて評価する。

次に、ウィンドウサイズを変えて実験する。
ウィンドウサイズの縦横比が保存時と再開時で異なる場合でも、スクロール位置が正しく復元するか検証する。

さらに、画面の向きを変えてテストする。
現在のモバイル端末では、画面の向きを変更できる。
保存時と再開時の端末の向きが異なる場合でも、スクロール位置を正しく復元できるか評価する。

異なるブラウザ間での復元について評価する。
本システムはPC上のChromeおよびiOS端末上のSafariのみに対応している。
そのため、使用するブラウザはChrome/Safariのみである。

異なる端末間での復元について検証する。
実験ではiPhone12\cite{iphone12}およびiPad(10.2インチ)\cite{ipad}
なお、PCとモバイル端末間での比較は、Chrome/Safari間の実験で実施する。

本実験で用いる条件と実験パターンを表\ref{tb:evl-scroll-position}にまとめる。

% textlint-disable
\begin{table}[htbp]
  \label{tb:evl-scroll-position}
  \caption{実験する条件}
  \begin{center}
    \begin{tabular}{|l|l|}
    \hline
    テスト条件  \\ \hline
    再開時のウィンドウサイズが保存時より縦に長い  \\ \hline
    再開時のウィンドウサイズが保存時より縦に短い  \\ \hline
    再開時のウィンドウサイズが保存時より横に長い  \\ \hline
    再開時のウィンドウサイズが保存時より横に短い  \\ \hline
    再開時のウィンドウサイズが保存時より縦横に長い  \\ \hline
    再開時のウィンドウサイズが保存時より縦横に短い  \\ \hline
    保存時は画面の向きが縦/再開時は画面の向きが横 \\ \hline
    保存時は画面の向きが横/再開時は画面の向きが縦 \\ \hline
    保存時はモバイル端末(iPhone12)/再開時はタブレット(iPad) \\ \hline
    保存時はタブレット(iPad)/再開時はモバイル端末(iPhone12) \\ \hline
    保存時はiOSのSafari/再開時はPCのChrome \\ \hline
    保存時はPCのChrome/再開時はiOSのSafari \\ \hline
    保存時と再開時の条件が同じ(Chrome拡張機能) \\ \hline
    保存時と再開時の条件が同じ(iOSアプリケーション) \\ \hline
    \end{tabular}
  \end{center}
\end{table}
% textlint-enable

% 実験に用いるWebページ
評価には複数の種類のWebサイトを用いる。
具体的には、一般的なWebページに加えて、1ページが長い(一番下までスクロールするために5秒以上かかる)Webページで検証する。
加えて、アニメーションが多用されているWebページ、SPA、その他の特殊なWebページをテストに使用する。
実験に用いるWebサイトを表\ref{tb:evl-basic-web-contents}に示す。

% textlint-disable
\begin{table}[htbp]
  \label{tb:evl-basic-web-contents}
  \caption{実験に用いるWebページ}
  \begin{center}
    \begin{tabular}{|l|l|}
    \hline
    種類 & WebページのURL  \\ \hline
    一般的なWebページ & \url{www.kgri.keio.ac.jp/research-synergies/talk-8.html} \\ \hline
    1ページが長いWebページ & \url{railstutorial.jp/chapters/beginning?version=5.1} \\ \hline
    アニメーションを多用しているWebページ & \url{ja.wikipedia.org/wiki/村井純} \\ \hline
    SPA & \url{plus-c.chuo-u.ac.jp/about/} \\ \hline
    特殊なWebサイト & \url{www.google.co.jp/maps/?hl=ja} \\ \hline
    \end{tabular}
  \end{center}
\end{table}
% textlint-enable