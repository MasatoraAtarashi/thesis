\chapter{評価}
\label{chap:evaluation}

本章では、ESP32マイコンとPC上で同一のWebAssemblyプログラムを実行することで、それぞれの環境における実行性能を比較し、マイコンにおけるWebAssembly実行環境の性能を考察する。

\section{評価手法}

\subsection{WebAssemblyバイナリ}
\label{subsec:wasm}

各実行環境で実行する同一のWebAssemblyプログラムとして、与えられた整数$n$に対して$n+1$番目のフィボナッチ数を返す関数$fib(n)$をC言語で実装した。

\begin{itembox}[l]{フィボナッチ関数のC言語による実装}
  \begin{verbatim}
    int fib(int n) {
      if (n <= 1) return 1;
      return fib(n - 1) + fib(n - 2);
    }
  \end{verbatim}
\end{itembox}

この関数のコンパイルには、LLVM/Clang 7.0.1を用いた。
LLVMのWebAssembly対応は試験的なものであるため、LLVMおよびClangコンパイラ自体を{\tt LLVM\_EXPERIMENTAL\_TARGETS\_TO\_BUILD}フラグに{\tt WebAssembly}を指定してソースコードからビルドした。

このLLVM/Clangを用いて{\tt fib}関数をコンパイルし、85バイトのWebAssemblyバイナリを得た。
WebAssemblyバイナリのコンパイル時に設定したオプションを表\ref{tab:compiler}に示す。

\begin{table}[htbp]
  \label{tab:compiler}
  \caption{WebAssemblyバイナリのコンパイル時に指定したオプション}
  \begin{center}
    \begin{tabular}{ll}
    \hline
    \verb|--target=wasm32-unknown-unknown-wasm| & WebAssemblyバイナリ \\
    & (wasm32)を生成する \\ \hline
    \verb|-s| & バイナリサイズを最適化し、 \\
    & デバッグ情報などを含めない \\ \hline
    \verb|-O3| & プログラムを最大レベルで最適化する \\ \hline
    \verb|-nostdlib| \verb|-nostartfiles| & 標準起動ファイルや\\
    & 標準ライブラリを使用しない \\ \hline
    \verb|--fvisibility=default| & 全ての関数をモジュール外 \\
    & から参照可能にする \\ \hline
    \verb|-Wl,--no-entry| & エントリーポイントが存在 \\
    & しないことを警告しない \\ \hline
    \end{tabular}
  \end{center}
\end{table}

\section{実行時間}

本項では、各環境ごとの実行時間の差を比較し考察する。
実行時間は、\ref{subsec:wasm}項の方法によりコンパイルしたWebAssemblyバイナリを、\ref{chap:implementation}章で述べた各環境上のホストプログラムで実行し、0から30までの$n$を引数として\verb|fib|関数を呼び出し、それぞれについて計測を行った。
ここでは、もっとも実行時間の掛かる$n=30$の場合に着目して比較する。

ESP32環境でのWebAssemblyプログラムの実行は、JIT環境での実行に対して、実行時間による比較で約40万倍の時間が掛かった。
また、実行時間にそれぞれの環境におけるCPUのクロック周波数を乗じ、クロック数による比較をしたところ、約32000倍のクロック数が必要だった。

\begin{table}[htbp]
  \caption{libwasmを用いた{\tt fib}関数の実行時間と、必要なクロック数の比}
  \label{tab:fib_time}
  \begin{center}
    \begin{tabular}{rrrrr}
      \hline
      & JIT環境での & ESP環境 & & \\
      n & 実行時間 ($\mu$s) & での実行時間 ($\mu$s) & 実行時間の比 & クロック数の比 \\ \hline \hline
      % 0  & 0     & 267           &            &           \\ \hline
      % 1  & 0     & 259           &            &           \\ \hline
      % 2  & 0     & 916           &            &           \\ \hline
      % 3  & 0     & 1,564         &            &           \\ \hline
      % 4  & 0     & 2,799         &            &           \\ \hline
      % 5  & 0     & 4,742         &            &           \\ \hline
      % 6  & 0     & 7,994         &            &           \\ \hline
      % 7  & 0     & 13,302        &            &           \\ \hline
      % 8  & 0     & 22,046        &            &           \\ \hline
      % 9  & 0     & 36,400        &            &           \\ \hline
      % 10 & 0     & 59,973        &            &           \\ \hline
      % 11 & 0     & 98,676        &            &           \\ \hline
      % 12 & 1     & 162,208       & 162,208.00 & 12,976.64 \\ \hline
      % 13 & 1     & 266,487       & 266,487.00 & 21,318.96 \\ \hline
      % 14 & 2     & 437,590       & 218,794.83 & 17,503.59 \\ \hline
      % 15 & 3     & 718,329       & 239,443.00 & 19,155.44 \\ \hline
      % 16 & 4     & 1,178,811     & 294,702.75 & 23,576.22 \\ \hline
      % 17 & 6     & 1,933,996     & 322,332.67 & 25,786.61 \\ \hline
      % 18 & 10    & 3,172,256     & 317,225.60 & 25,378.05 \\ \hline
      % 19 & 15    & 5,202,283     & 346,818.87 & 27,745.51 \\ \hline
      % 20 & 24    & 8,529,805     & 355,408.54 & 28,432.68 \\ \hline
      % 21 & 39    & 13,983,115    & 358,541.42 & 28,683.31 \\ \hline
      % 22 & 63    & 22,918,957    & 363,792.97 & 29,103.44 \\ \hline
      % 23 & 102   & 37,558,837    & 368,223.90 & 29,457.91 \\ \hline
      % 24 & 168   & 61,540,310    & 366,311.37 & 29,304.91 \\ \hline
      % 25 & 272   & 100,818,119   & 370,654.85 & 29,652.39 \\ \hline
      % 26 & 431   & 165,139,586   & 383,154.49 & 30,652.36 \\ \hline
      % 27 & 697   & 270,457,743   & 388,031.20 & 31,042.50 \\ \hline
      % 28 & 1,127 & 442,878,521   & 392,971.18 & 31,437.69 \\ \hline
      % 29 & 1,818 & 725,117,084   & 398,854.28 & 31,908.34 \\ \hline
      30 & 2,941 & 1,187,057,130 & 403,623.64 & 32,289.89 \\ \hline
    \end{tabular}
  \end{center}
\end{table}

次に、macOS環境とESP32環境における実行時間を比較することで、異なる環境下で同一のインタプリタを用いた際の実行時間の変化を求めた。その結果$n=30$の場合において、ESP32環境はmacOS環境の約15倍のクロック数を必要とする事がわかった。

\begin{table}[htbp]
  \caption{libwasmを用いた{\tt fib}関数の実行時間と、必要なクロック数の比}
  \label{tab:fib_time}
  \begin{center}
    \begin{tabular}{rrrrr}
      \hline
         & macOS環境での & ESP32環境での & 実行に必要な \\
       n & 実行時間($\mu$s) & 実行時間($\mu$s) & クロック数の比 \\ \hline \hline
      %  0 &         3 &           267 &  7.1 倍 \\ \hline
      %  1 &         3 &           259 &  6.9 倍 \\ \hline
      %  2 &         9 &           916 &  7.8 倍 \\ \hline
      %  3 &        15 &         1,564 &  8.1 倍 \\ \hline
      %  4 &        25 &         2,799 &  8.9 倍 \\ \hline
      %  5 &        42 &         4,742 &  9.0 倍 \\ \hline
      %  6 &        70 &         7,994 &  9.0 倍 \\ \hline
      %  7 &       118 &        13,302 &  9.0 倍 \\ \hline
      %  8 &       161 &        22,046 & 10.9 倍 \\ \hline
      %  9 &       287 &        36,400 & 10.1 倍 \\ \hline
      % 10 &       408 &        59,973 & 11.7 倍 \\ \hline
      % 11 &       648 &        98,676 & 12.1 倍 \\ \hline
      % 12 &     1,432 &       162,208 &  9.0 倍 \\ \hline
      % 13 &     2,354 &       266,487 &  9.0 倍 \\ \hline
      % 14 &     3,642 &       437,590 &  9.6 倍 \\ \hline
      % 15 &     5,022 &       718,329 & 11.4 倍 \\ \hline
      % 16 &     8,608 &     1,178,811 & 10.9 倍 \\ \hline
      % 17 &    13,815 &     1,933,996 & 11.2 倍 \\ \hline
      % 18 &    22,328 &     3,172,256 & 11.3 倍 \\ \hline
      % 19 &    35,609 &     5,202,283 & 11.6 倍 \\ \hline
      % 20 &    52,961 &     8,529,805 & 12.8 倍 \\ \hline
      % 21 &    83,772 &    13,983,115 & 13.3 倍 \\ \hline
      % 22 &   134,115 &    22,918,957 & 13.6 倍 \\ \hline
      % 23 &   216,433 &    37,558,837 & 13.8 倍 \\ \hline
      % 24 &   349,573 &    61,540,310 & 14.0 倍 \\ \hline
      % 25 &   562,661 &   100,818,119 & 14.3 倍 \\ \hline
      % 26 &   909,827 &   165,139,586 & 14.5 倍 \\ \hline
      % 27 & 1,455,711 &   270,457,743 & 14.8 倍 \\ \hline
      % 28 & 2,353,144 &   442,878,521 & 15.0 倍 \\ \hline
      % 29 & 3,847,535 &   725,117,084 & 15.0 倍 \\ \hline
      30 & 6,237,998 & 1,187,057,130 & 15.2 倍 \\ \hline
    \end{tabular}
  \end{center}
\end{table}

この結果をJIT環境における実行時間の値に適用したところ、ESP32においてJITによる高速化が得られた場合、$n=30$の時に約560ミリ秒の実行時間となることが推測された。
この結果から、マイコンにおいてJITによる高速化を行った場合、高速な演算性能が求められる処理でなければ実用的な速度で実行できる事が示唆された。

\begin{table}[htbp]
  \caption{Webブラウザにおける実行時間と、推測されるESP32上での実行時間}
  \label{tab:fib_time_browser}
  \begin{center}
    \begin{tabular}{rrr}
      \hline
         & JIT環境での & ESP32環境での \\
       n & 実行時間($\mu$s) & 実行時間(推測、$\mu$s) \\ \hline \hline
      % 12 &     1 &     113 \\ \hline
      % 13 &     1 &     113 \\ \hline
      % 14 &     2 &     240 \\ \hline
      % 15 &     3 &     429 \\ \hline
      % 16 &     4 &     548 \\ \hline
      % 17 &     6 &     840 \\ \hline
      % 18 &    10 &   1,421 \\ \hline
      % 19 &    15 &   2,191 \\ \hline
      % 20 &    24 &   3,865 \\ \hline
      % 21 &    39 &   6,510 \\ \hline
      % 22 &    63 &  10,766 \\ \hline
      % 23 &   102 &  17,701 \\ \hline
      % 24 &   168 &  29,575 \\ \hline
      % 25 &   272 &  48,737 \\ \hline
      % 26 &   431 &  78,229 \\ \hline
      % 27 &   697 & 129,496 \\ \hline
      % 28 & 1,127 & 212,109 \\ \hline
      % 29 & 1,818 & 342,625 \\ \hline
      30 & 2,941 & 559,656 \\ \hline
    \end{tabular}
  \end{center}
\end{table}

\section{メモリフットプリント}

ESP32環境での実行について、メモリフットプリントとしてヒープ領域に確保されるメモリの最大量を計測した。
なお、スタックサイズは150000バイトに設定した。
引数\verb|n|を0から13まで変化させた際の、関数実行のメモリフットプリントの推移を表\ref{tab:heap_size}に示す。

定数を返すのみである$n=0$および$n=1$の時、メモリフットプリントは540バイトで、実行速度と同様に$n$による変化は無かった。
再帰的な関数呼び出しが発生する$n=2$以降は、208バイトずつ上昇していった。
これは、WebAssembly実行環境のスタックにおけるメモリ消費であると考えられる。

150キロバイト程度の未確保領域があることを踏まえると、関数実行における280バイトのオーバーヘッドは実用的な範囲として許容できると考えられる。

\begin{table}[htbp]
  \caption{ESP32環境におけるメモリフットプリントの推移}
  \label{tab:heap_size}
  \begin{center}
    \begin{tabular}{rrrr}
      \hline
      & 未確保領域 & 起動からの確保量 & 変化量 \\
      & (バイト) & (バイト) & (バイト) \\ \hline \hline
      起動後      & 152,636 & - & - \\ \hline
      パース後     & 152,600 & 36 & 36 \\ \hline
      インスタンス化後 & 152,440  & 196 & 160 \\ \hline
      $n=0$  & 152,096 &   540 & 344 \\ \hline
      1  & 152,096 &   540 &   0 \\ \hline
      2  & 151,816 &   820 & 280 \\ \hline
      3  & 151,536 & 1,100 & 280 \\ \hline
      % 4  & 151,256 & 1,380 & 280 \\ \hline
      % 5  & 150,976 & 1,660 & 280 \\ \hline
      % 6  & 150,696 & 1,940 & 280 \\ \hline
      % 7  & 150,416 & 2,220 & 280 \\ \hline
      % 8  & 150,136 & 2,500 & 280 \\ \hline
      % 9  & 149,856 & 2,780 & 280 \\ \hline
      % 10 & 149,576 & 3,060 & 280 \\ \hline
      % 11 & 149,296 & 3,340 & 280 \\ \hline
      % 12 & 149,016 & 3,620 & 280 \\ \hline
      % 13 & 148,736 & 3,900 & 280 \\ \hline
      % 14 & 148,456 & 4,180 & 280 \\ \hline
      % 15 & 148,176 & 4,460 & 280 \\ \hline
      % 16 & 147,896 & 4,740 & 280 \\ \hline
      % 17 & 147,616 & 5,020 & 280 \\ \hline
      % 18 & 147,336 & 5,300 & 280 \\ \hline
      % 19 & 147,056 & 5,580 & 280 \\ \hline
      % 20 & 146,776 & 5,860 & 280 \\ \hline
      % 21 & 146,496 & 6,140 & 280 \\ \hline
      % 22 & 146,216 & 6,420 & 280 \\ \hline
      % 23 & 145,936 & 6,700 & 280 \\ \hline
      % 24 & 145,656 & 6,980 & 280 \\ \hline
      % 25 & 145,376 & 7,260 & 280 \\ \hline
      % 26 & 145,096 & 7,540 & 280 \\ \hline
      % 27 & 144,816 & 7,820 & 280 \\ \hline
      % 28 & 144,536 & 8,100 & 280 \\ \hline
      % 29 & 144,256 & 8,380 & 280 \\ \hline
      \vdots & & & \\ \hline
      30 & 143,976 & 8,660 & 280 \\ \hline
    \end{tabular}
  \end{center}
\end{table}
