\begin{eabstract}

The recent performance improvement and commoditization of embedded devices
allow everything around us to be connected to the Internet, and made Internet
of Things (IoT) popular.

ESP32, developed by Espressif Systems, is one kind of these devices with
relatively high performance and low price.
An ESP32 chip have a dual-core CPU and $>$500KB of RAM, with Wi-Fi, Bluetooth
and Bluetooth Low Energy connectivity.

Although ESP32 made it easier to design Internet-connected hardware, you have to
choose from a limited number of development enviroments which support Xtensa
architecture, the instruction set architecture of the CPU of ESP32, to build
software to run on it.
This can be strong limitations of programming languages and development tools for
developers.

In this research, for the purpose of making more development enviroments
available for software development for ESP32, we implemented a runtime of
WebAssembly, a virtual instruction set which began to be supported by several
tools.
Also, we ensured a WebAssembly program generated by a general compiler can get
run on ESP32.
Additionally, we measured and compared the execution time and memory footprint
of a program on and not on our runtime.

As a result, while we discovered that there are 3x-10x of overhead of execution
time, and 2x-5x of memory footprint, we confirmed we can use WebAssembly on ESP32
with just reasonable overhead.
Consequently, we confirmed we can extend the choise of development enviroments
of ESP32 software by implementing a WebAssembly runtime.

\end{eabstract}
