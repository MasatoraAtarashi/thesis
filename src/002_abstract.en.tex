\begin{eabstract}

  Microcontrollers statically execute their program written on its RAM such as a flash memory, loading it by the boot loader at startup.
  Consequently, in order to change what to run, restarting after getting the program on the RAM rewritten by connecting it directly, or doing OTA (Over-the-air) update over a wireless connection.

  On the other hand, Web applications load and execute programs dynamically.
  As an application platform, the Web browser follows links on a Web page and executes downloaded programs.
  That is why different applications can be provided for each Web page and programs can be changed depending on the user environment.

  We propose the design of an application platform which uses WebAssembly as the executable format to make dynamic execution possible on microcontrollers.
  WebAssembly is a virtual instruction set architecture designed to be run on Web browsers and to speed up the dynamic execution on Web browsers.
  Although it is mainly aimed at being executed on Web browsers, its specification has no restriction nor limit against being used outside Web browsers.

  In this research, we designed a WebAssembly runtime for microcontrollers which dynamically downloads and runs WebAssembly programs.
  And we implemented a WebAssembly interpreter in C language, in order to evaluate the feasibility of this runtime.

  We executed a single WebAssembly binary on an ESP32 microcontroller and on a PC with Intel Core i5 3GHz CPU.
  As a result, we found that the ESP32 requires 7 to 15 times as many clocks as the PC.
  Applying this result to the execution time on Safari Web browser to run the same WebAssembly program on the PC, it was estimated that a WebAssembly runtime which has execution efficiency as Web browsers will be able to calculate 30th Fibonacci number in 560 msec approximately.

\end{eabstract}
