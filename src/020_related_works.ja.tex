\chapter{関連技術}
\label{chap:related_works}

本章では、既存のブックマークシステムを整理し、本研究の優位性について説明する。

\section{既存のブックマークシステム}

\subsection{ブラウザの提供するブックマーク機能}
SafariやChromeなどの標準的なブラウザは、2種類のブックマーク機能を提供している。

1つ目は、ショートカット機能である。
本機能は、任意のWebページへのショートカットをブラウザ内に保存する機能である。
保存したショートカットはブラウザのトップ画面やナビゲーションバーなどに表示される。
本機能は頻繁に使用するWebページへ高速にアクセスする目的で使用される。

2つ目は、リーディングリスト機能である。
本機能は、後で読みたいWebページを保存しておくために使用する。
ブックマーク機能との違いは、リーディングリストに追加されるWebページは頻繁に使用されるわけではない点である。
そのため、ユーザがリーディングリストに追加したWebページを閲覧する際には、ページのどの部分を見たかったか、などの記憶が失われている可能性が高い。

これらの機能では、単にWebページのURLを保存する。
そのため、保存時の閲覧状態は失われてしまう。
また、データをブラウザ内に保存するため、異なるブラウザ間では使用できない。

Safariにおいてブックマーク機能を利用している画面を図\ref{fig:safari-bookmark}に示す。

\begin{figure}[htbp]
  \label{fig:safari-bookmark}
  \begin{center}
    \includegraphics[bb=0 0 585 1266,width=5cm]{img/020_related_works/safari-bookmark.pdf}
  \end{center}
  \caption{Safariのブックマーク機能}
\end{figure}

\subsection{ブックマーク機能を提供するWebサービス}
ブラウザの提供するブックマーク機能では、異なるブラウザ間でブックマークを共有できないという欠点があった。
その課題を解決するために誕生したのが、Webサービスとしてのブックマーク機能である。
これらのWebサービスでは、ユーザが保存したブックマークをオンライン上で保存する。
それによって、ユーザはブラウザや端末を跨いでブックマークを利用できる。
ブックマーク機能を提供するWebサービスのうち、著名なものを紹介する。

\subsubsection{Pocket}
PocketはMozilla\cite{mozilla}が提供しているサービスである。
Pocketでは複数の端末やブラウザからブックマークを利用できる一方、保存時の閲覧状態は失われてしまう。
Pocketのサービス画面を図\ref{fig:pocket}に示す。

\begin{figure}[htbp]
  \label{fig:pocket}
  \begin{center}
    \includegraphics[bb=0 0 545 378,width=15cm]{img/020_related_works/pocket.pdf}
  \end{center}
  \caption{Pocket}
\end{figure}

\subsubsection{Instapaper}
Instapaper\cite{instapaper}はInstant Paper, Inc.\cite{instant-paper-inc}が提供しているサービスである。
このサービスの特徴は、ブックマークをオフラインで利用できる点である。
Instapaperでも、Pocket同様に保存時の閲覧状態は失われてしまう。
Instapaperのサービス画面を図\ref{fig:instapaper}に示す。

\begin{figure}[htbp]
  \label{fig:instapaper}
  \begin{center}
    \includegraphics[bb=0 0 640 400,width=15cm]{img/020_related_works/instapaper.pdf}
  \end{center}
  \caption{Instapaper}
\end{figure}

\subsection{ソーシャルブックマーク}
上記で挙げた例以外に、ソーシャルブックマークというブックマークシステムも存在する。
ソーシャルブックマークとは、インターネット上で自分のブックマークを公開し、他のユーザと共有するサービスである。
ソーシャルブックマークの例として、はてな社\cite{hatena}が提供するはてなブックマーク\cite{hatena-bookmark}というサービスがある。
インターネット上でブックマークを共有するというサービスの特性上、閲覧状態は保存されない。
はてなブックマークのサービス画面を図\ref{fig:hatena-bookmark}に示す。

\begin{figure}[htbp]
  \label{fig:hatena-bookmark}
  \begin{center}
    \includegraphics[bb=0 0 1020 940,width=15cm]{img/020_related_works/hatena-bookmark.pdf}
  \end{center}
  \caption{はてなブックマーク}
\end{figure}

以上から、既存のブックマークシステムはいずれも保存時の閲覧状態が失われてしまうことがわかった。
本研究で提案するシステムでは、ブックマーク時の閲覧状態を保存・復元できる。
加えて、Webページ内のリソースを指定してブックマークできるため、より高速に目当ての情報にアクセス可能なショートカット機能としても使用できる。

本章では、既存のブックマークシステムについて整理し、本研究の優位性を説明した。
次章では、本研究で提案するWeb-Snapshotシステムについて述べる。