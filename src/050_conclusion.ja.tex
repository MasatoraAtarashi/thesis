\chapter{結論}
\label{chap:conclusion}

\section{本研究のまとめ}
\label{section:matome}

本研究では、ESP32で汎用的な開発環境を用いることを目的として、仮想命令セットアーキテクチャとして
のWebAssemblyについて実行環境を実装し、既存の開発環境を用いた場合と比較した。
単純な関数をWebAssembly実行環境で実行した結果、実行速度において約3倍〜10倍、
メモリフットプリントにおいて約2倍〜5倍のオーバーヘッドがあることがわかった。
このことから、ESP32において現実的なオーバーヘッドでWebAssemblyバイナリを実行できることがわかった。

ただし、本研究で実装したWebAssembly実行環境は、WebAssemblyプログラムから
OSやハードウェアの機能(標準入出力やGPIOへのアクセスなど)を用いることができない。
そのため、本実行環境で実行可能なプログラムは、単純な数値演算の範囲にとどまる。
したがって、暗号アルゴリズムや特定のファイルフォーマットのパースといった、
既に何かしらのプログラミング言語における実装が存在する処理を再実装することなく用いる、
といったユースケースが考えられる。

\section{今後の課題}

本研究では、WebAssembly実行環境は他のCプログラムに内包されて用いられることを想定して実装した。
ESP32上で動作するプログラムを全てWebAssemblyで記述するには、実行環境がWebAssemblyプログラムに対して
ハードウェアの管理などのOSに近い機能を提供する必要があるだろう。

また、本実行環境を実用的なものとするには、未だWebAssemblyプログラムの実行効率に最適化の余地がある。
例えば、Just-in-Timeコンパイルといった高速化の手法を適用することはできなかった。
仮想命令セットアーキテクチャを高速に実行するための手法は広く研究の対象となっているため、
本環境下における有効性を検証していく価値があるだろう。
