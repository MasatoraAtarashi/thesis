\chapter{結論}
\label{chap:conclusion}

\section{本研究のまとめ}

本研究では、ESP32で汎用的な開発環境を用いることを目的として、仮想命令セットアーキテクチャとして
のWebAssemblyについて実行環境を実装し、既存の開発環境を用いた場合と比較した。
単純な関数をWebAssembly実行環境で実行した結果、実行速度において約$n$\%、
メモリフットプリントにおいて約$m$\%のオーバーヘッドがあることがわかった。
このことから、ESP32において現実的なオーバーヘッドでWebAssemblyバイナリを実行できることがわかった。

\section{今後の課題と展望}
