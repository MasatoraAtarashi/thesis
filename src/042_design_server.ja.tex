\section{サーバ側設計}
本節では、サーバのデータベース設計およびAPI設計について説明する。加えて、PDFからページ数を抽出するモジュールの設計について述べる。

\subsection{データベース設計}
本システムのデータベースにはユーザの認証情報を格納するUserテーブル。
ユーザがブックマークしたWebコンテンツを保存するContentテーブル。
ユーザが作成したフォルダ情報を格納するFolderテーブル。
コンテンツとフォルダを紐付けるためのContentFolderテーブル、の4つのテーブルが存在する。
Userテーブルのカラム名・データ型・値の説明を表\ref{tb:design-user-table}に示す。
Contentテーブルを表\ref{tb:design-content-table}に示す。
Folderテーブルを表\ref{tb:design-folder-table}に示す。
ContentFolderテーブルを表\ref{tb:design-content-folder-table}に示す。

% textlint-disable
\begin{table}[htbp]
  \label{tb:design-user-table}
  \caption{Userテーブル}
  \begin{center}
    \begin{tabular}{|l|l|l|}
      \hline
      \multicolumn{1}{|c|}{\textbf{カラム名}} & \multicolumn{1}{|c|}{\textbf{データ型}} & \multicolumn{1}{|c|}{\textbf{値の説明}} \\\hline
      id & bigint & ユーザの識別子 \\\hline
      email & varchar & メールアドレス \\\hline
      encrypted\_password & varchar & 暗号化したパスワード \\\hline
      reset\_password\_token & varchar & パスワードリセット時に用いる時限トークン \\\hline
      reset\_password\_sent\_at & timestamp & パスワードリセットメールを送った日時 \\\hline
      remember\_created\_at & timestamp & 最後のログインした日時 \\\hline
      lambda & boolean & PDFのページ数を更新するLambda用権限を有しているか \\\hline
      created\_at & timestamp & 会員登録日時 \\\hline
      updated\_at & timestamp & ユーザ情報更新日時 \\\hline
      provider & varchar & 認証に利用した外部サービス \\\hline
      uid & varchar & 認証に用いるユーザ識別子 \\\hline
      tokens & text & 認証に用いるトークン \\\hline
    \end{tabular}
  \end{center}
\end{table}

\begin{table}[htbp]
  \label{tb:design-content-table}
  \caption{Contentテーブル}
  \begin{center}
    \begin{tabular}{|l|l|l|}
      \hline
      \multicolumn{1}{|c|}{\textbf{カラム名}} & \multicolumn{1}{|c|}{\textbf{データ型}} & \multicolumn{1}{|c|}{\textbf{値の説明}} \\\hline
      id & bigint & ブックマークの識別子 \\\hline
      user\_id & bigint & ブックマークを保存したユーザの識別子 \\\hline
      title & varchar & タイトル \\\hline
      content\_type & varchar & ブックマークの種類(WEB,動画,PDFなど) \\\hline
      url & varchar & URL \\\hline
      sharing\_url & varchar & 共有用URL \\\hline
      file\_url & varchar & ファイルの保存先URL \\\hline
      thumbnail\_img\_url & varchar & サムネイル画像のURL \\\hline
      scroll\_position\_x & integer & 保存時のスクロール位置(横) \\\hline
      scroll\_position\_y & integer & 保存時のスクロール位置(縦) \\\hline
      window\_outer\_width & integer & 保存時のブラウザウィンドウの外側の幅 \\\hline
      window\_outer\_height & integer & 保存時のブラウザウィンドウの外側の高さ \\\hline
      video\_playback\_position & integer & 保存時の動画再生位置 \\\hline
      audio\_playback\_position & integer & 保存時の音声再生位置 \\\hline
      liked & boolean & お気に入り \\\hline
      pdf & string & PDFのスクリーンショットを保存しているストレージのURL \\\hline
      delete\_flag & boolean & 論理削除フラグ \\\hline
      created\_at & timestamp & 作成日時 \\\hline
      updated\_at & timestamp & 更新日時 \\\hline
      deleted\_at & timestamp & 論理削除日時 \\\hline
    \end{tabular}
  \end{center}
\end{table}

\begin{table}[htbp]
  \label{tb:design-folder-table}
  \caption{Folderテーブル}
  \begin{center}
    \begin{tabular}{|l|l|l|}
      \hline
      \multicolumn{1}{|c|}{\textbf{カラム名}} & \multicolumn{1}{|c|}{\textbf{データ型}} & \multicolumn{1}{|c|}{\textbf{値の説明}} \\\hline
      id & bigint & フォルダの識別子 \\\hline
      name & varchar & フォルダ名 \\\hline
      user\_id & bigint & フォルダを作成したユーザの識別子 \\\hline
    \end{tabular}
  \end{center}
\end{table}

\begin{table}[htbp]
  \label{tb:design-content-folder-table}
  \caption{ContentFolderテーブル}
  \begin{center}
    \begin{tabular}{|l|l|l|}
      \hline
      \multicolumn{1}{|c|}{\textbf{カラム名}} & \multicolumn{1}{|c|}{\textbf{データ型}} & \multicolumn{1}{|c|}{\textbf{値の説明}} \\\hline
      id & bigint & コンテンツとフォルダのペアの識別子 \\\hline
      content\_id & bigint & コンテンツの識別子 \\\hline
      folder\_id & bigint & フォルダの識別子 \\\hline
    \end{tabular}
  \end{center}
\end{table}
% textlint-enable

\subsection{API設計}
本システムでは、認証関連API・コンテンツ関連API・フォルダ関連APIの3種類のWebAPIを提供している。
本APIはREST\footnote{REpresentational State Transfer。Roy Fielding氏が2000年に発表したWebAPIのインターフェイスの設計アーキテクチャ}の思想に基づいて設計しており、HTTPリクエストを通じて呼び出すことができる。

認証関連APIは、ユーザの認証に関連するAPIを提供する。
認証関連APIが提供する各APIのエンドポイントと概要を表\ref{tb:design-auth-api}に示す。

% textlint-disable
\begin{table}[htbp]
  \label{tb:design-auth-api}
  \caption{認証関連APIエンドポイント一覧}
  \begin{center}
    \begin{tabular}{|l|l|l|}
      \hline
      \multicolumn{1}{|c|}{\textbf{HTTPメソッド}} & \multicolumn{1}{|c|}{\textbf{パス}} & \multicolumn{1}{|c|}{\textbf{概要}} \\\hline
      POST & /v1/auth & 会員登録 \\\hline
      POST & /v1/auth/sign\_in & ログイン \\\hline
      GET & /v1/auth/google & Googleアカウントで認証 \\\hline
      GET & /v1/auth/google/callback & Googleの認証後にリダイレクトされるエンドポイント \\\hline
      GET & /v1/auth/twitter & Twitterアカウントで認証 \\\hline
      GET & /v1/auth/twitter/callback & Twitterの認証後にリダイレクトされるエンドポイント \\\hline
      GET & /v1/auth/github & GitHubアカウントで認証 \\\hline
      GET & /v1/auth/github/callback & GitHubの認証後にリダイレクトされるエンドポイント \\\hline
    \end{tabular}
  \end{center}
\end{table}
% textlint-enable

上記のうち、会員登録するためのリクエストおよびレスポンスの例を以下に示す。

% textlint-disable
\begin{itembox}[l]{会員登録リクエストの例}
  \label{auth-request-curl}
  \begin{verbatim}
    curl -X POST --header 'Content-Type: application/json'
     --header 'Accept: application/json' -d '{
      "password": "password",
      "password_confirmation": "password_confirmation",
      "email": "email"
    }' 'https://web-shiori.herokuapp.com/v1/auth'
  \end{verbatim}
\end{itembox}

\begin{itembox}[l]{会員登録リクエストのレスポンス例}
  \label{auth-response-json}
  \begin{verbatim}
    200 OK
    {
      "status": "success",
      "data": {
        "id": 126,
        "provider": "email",
        "uid": "example@example.com",
        "allow_password_change": true,
        "email": "example2@example.com",
        "lambda": false,
        "created_at": "2019-05-12T20:48:24.000+09:00",
        "updated_at": "2019-05-12T20:48:24.000+09:00"
      }
    }
  \end{verbatim}
\end{itembox}
% textlint-enable

コンテンツ関連APIは、ブックマークを操作するためのAPIを提供する。
コンテンツ関連APIが提供する各APIのエンドポイントと概要を表\ref{tb:design-content-api}に示す。
上記のうち、Webページをブックマークするためのリクエストおよびレスポンスの例を以下に示す。

% textlint-disable
\begin{table}[htbp]
  \label{tb:design-content-api}
  \caption{コンテンツ関連APIエンドポイント一覧}
  \begin{center}
    \begin{tabular}{|l|l|l|}
      \hline
      \multicolumn{1}{|c|}{\textbf{HTTPメソッド}} & \multicolumn{1}{|c|}{\textbf{パス}} & \multicolumn{1}{|c|}{\textbf{概要}} \\\hline
      GET & /v1/content & ブックマークしたコンテンツ一覧取得 \\\hline
      POST & /v1/content & Webページをブックマークする \\\hline
      PUT & /v1/content/\{content\_id\} & ブックマークを修正する \\\hline
      DELETE & /v1/content/\{content\_id\} & ブックマークを解除する \\\hline
    \end{tabular}
  \end{center}
\end{table}
% textlint-enable

% textlint-disable
\begin{itembox}[l]{コンテンツをブックマークするリクエストの例}
  \label{content-request-curl}
  \begin{verbatim}
    curl -X POST --header 'Content-Type: application/json' 
    --header 'Accept: application/json' 
    --header 'access-token: access-token' 
    --header 'client: client' 
    --header 'uid: uid' -d '{
      "title": "Web-Shioriデモ動画",
      "url": "https://www.youtube.com/watch?v=1DcjMwkmNvA",
      "video_playback_position": 50,
      "scroll_position_x": 0,
      "scroll_position_y": 200,
      "window_outer_width": 1440,
      "window_outer_height": 1080
    }' 'https://web-shiori.herokuapp.com/v1/content'
  \end{verbatim}
\end{itembox}

\begin{itembox}[l]{コンテンツをブックマークするリクエストのレスポンス例}
  \label{content-response-json}
  \begin{verbatim}
    200 OK
    {
      "data": {
        "id": 126,
        "content_type": "video",
        "title": "Web-Shioriデモ動画",
        "url": "https://www.youtube.com/watch?v=1DcjMwkmNvA",
        "sharing_url": "https://www.youtube.com/watch?v=1DcjMwkmNvA",
        "file_url": null,
        "thumbnail_img_url": "https://web-shiori.com/sample.jpg",
        "scroll_position_x": 0,
        "scroll_position_y": 500,
        "window_outer_width": 1440,
        "window_outer_height": 1080,
        "video_playback_position": 50,
        "audio_playback_position": 0,
        "liked": true,
        "delete_flag": false,
        "deleted_at": null,
        "created_at": "2019-05-12T20:48:24.000+09:00",
        "updated_at": "2019-05-12T20:48:24.000+09:00"
      }
    }
  \end{verbatim}
\end{itembox}
% textlint-enable

フォルダ関連APIは、フォルダに関連するAPIを提供する。
フォルダ関連APIが提供する各APIのエンドポイントと概要を表\ref{tb:design-folder-api}に示す。

% textlint-disable
\begin{table}[htbp]
  \label{tb:design-folder-api}
  \caption{フォルダ関連APIエンドポイント一覧}
  \begin{center}
    \begin{tabular}{|l|l|l|}
      \hline
      \multicolumn{1}{|c|}{\textbf{HTTPメソッド}} & \multicolumn{1}{|c|}{\textbf{パス}} & \multicolumn{1}{|c|}{\textbf{概要}} \\\hline
      GET & /v1/folder & フォルダ一覧取得 \\\hline
      POST & /v1/folder & フォルダ作成 \\\hline
      PUT & /v1/folder/\{folder\_id\} & フォルダ更新 \\\hline
      DELETE & /v1/folder/\{folder\_id\} & フォルダ削除 \\\hline
      GET & /v1/folder/\{folder\_id\}/content & フォルダ内コンテンツ一覧取得 \\\hline
      POST & /v1/folder/\{folder\_id\}/content/\{content\_id\} & フォルダにコンテンツを追加 \\\hline
      DELETE & /v1/folder/\{folder\_id\}/content/\{content\_id\} & フォルダからコンテンツを削除 \\\hline
    \end{tabular}
  \end{center}
\end{table}
% textlint-enable

\subsection{PDFからページ数を抽出するモジュール}
本項では、Webブラウザ上で閲覧中のPDFからページ数を抽出・保存するモジュールの設計について説明する。

Chromeが提供するPDFビューワには、ページ数を取得するためのAPIが存在しない。
そのため、スクロール位置などのようにJavaScriptプログラムを通じてデータを取得できない。

そこで、本システムではユーザのブラウザの画面のキャプチャを保存し、OCRを用いてそのキャプチャから閲覧中のPDFのページ数を取得する、という方法を提案する。

ChromeのPDFビューワは、図\ref{fig:pdf-viewer-nav-bar-page-num}のようにナビゲーションバーに現在ユーザが閲覧しているページ数を表示する。
画面のキャプチャからOCRを用いてこの数値を抽出することで、閲覧中のPDFのページ数を取得できる。
本機能の処理の流れを図\ref{fig:design-pdf-page-num-extract}に示す。

\begin{figure}[htbp]
  \label{fig:pdf-viewer-nav-bar-page-num}
  \begin{center}
    \includegraphics[bb=0 0 1440 900,width=15cm]{img/040_design/pdf-viewer-nav-bar-page-num.pdf}
  \end{center}
  \caption{PDFビューワのナビゲーションバーに表示されるページ数}
\end{figure}

\begin{figure}[htbp]
  \label{fig:design-pdf-page-num-extract}
  \begin{center}
    \includegraphics[bb=0 0 511 161,width=15cm]{img/040_design/design-pdf-page-num-extract.pdf}
  \end{center}
  \caption{PDFのページ数を取得する処理の流れ}
\end{figure}

サーバ側では、取得したページ数をURLのフラグメントに付与する。(例: \url{https://www.usenix.org/system/files/conference/hotcloud16/hotcloud16_burns.pdf#page=5})
クライアントではこのURLでページを開くことで、PDFのページ数を復元できる。
